\documentclass[10pt,a4paper,oneside,openright]{book}

% TODO
% among us

%%%%%%%%%%%%%%%%%%%%%%%%%%%%%%%%%%%%%%%%%
% Template Dispense
% Autore: Teo Bucci
%%%%%%%%%%%%%%%%%%%%%%%%%%%%%%%%%%%%%%%%%

%----------------------------------------------------------------------------------------
%	FONTS AND LANGUAGE
%----------------------------------------------------------------------------------------

\usepackage[T1]{fontenc} % Use 8-bit encoding that has 256 glyphs
\usepackage[utf8]{inputenc} % Required for including letters with accents
\usepackage[english]{babel} % Italian language/hyphenation

%----------------------------------------------------------------------------------------
%	PACKAGES
%----------------------------------------------------------------------------------------

\usepackage{amsmath,amssymb,amsthm} % amssymb carica anche amsfonts

\usepackage{dsfont} % per funzione indicatrice
\newcommand{\Ind}{\mathds{1}}
%\usepackage{tikz-among-us}

\usepackage{booktabs}
\usepackage{pgfplots}
\usepackage{tikz}
\usetikzlibrary{positioning,shapes.misc,intersections,shapes.symbols}
\usepackage{mathdots}
\usepackage{cancel}
\usepackage{color}
\usepackage{siunitx}
\usepackage{array}
\usepackage{multirow}
\usepackage{makecell}
\usepackage{tabularx}
%\usepackage{booktabs}
\usepackage{caption}\captionsetup{belowskip=12pt,aboveskip=4pt}
\usepackage{subcaption}
\usepackage{placeins} % The placeins package gives the command \FloatBarrier, which will make sure any floats will be put in before this point
\usepackage{flafter}  % The flafter package ensures that floats don't appear until after they appear in the code.

\usepackage{import}
\usepackage{pdfpages}
\usepackage{transparent}
\usepackage{xcolor}
\usepackage{graphicx}
\graphicspath{ {./images/} }
\usepackage{float}
\usepackage[italian]{varioref}

\newcommand{\fg}[3][\relax]{%
  \begin{figure}[H]%[htp]%
    \centering
    \captionsetup{width=0.7\textwidth}
      \includegraphics[width = #2\textwidth]{#3}%
      \ifx\relax#1\else\caption{#1}\fi
      \label{#3}
  \end{figure}%
  \FloatBarrier%
}

%----------------------------------------------------------------------------------------
%	PARAGRAFI, INTERLINEA E MARGINE
%----------------------------------------------------------------------------------------

\usepackage[none]{hyphenat} % Per non far andare a capo le parole con il trattino

\emergencystretch 3em % Per evitare che il testo vada oltre i margini

% \parindent 0ex % Toglie intendamento paragrafi, e' incluso nel pacchetto \parskip
% \setlength{\parindent}{4em} % Variante di quello sopra

% \setlength{\parskip}{\baselineskip} % Cambia spazio tra paragrafi (posso mettere anche 1em) inclusa la table of contents, pertanto uso il comando che segue

\usepackage[skip=0.2\baselineskip+2pt]{parskip}

% \renewcommand{\baselinestretch}{1.5} % Cambia interlinea

%----------------------------------------------------------------------------------------
%	HEADERS AND FOOTERS
%----------------------------------------------------------------------------------------

\usepackage{fancyhdr}


\fancypagestyle{toc}{%
\fancyhf{}%
\fancyfoot[C]{\thepage}%
\renewcommand{\headrulewidth}{0pt}%
\renewcommand{\footrulewidth}{0pt}
}

\fancypagestyle{fancy}{%
\fancyhf{}%
\fancyhead[RE]{\nouppercase{\leftmark}}%
\fancyhead[LO]{\nouppercase{\rightmark}}%
\fancyhead[LE,RO]{\thepage}%
\renewcommand{\footrulewidth}{0pt}%
\renewcommand{\headrulewidth}{0.4pt}
}

% Removes the header from odd empty pages at the end of chapters
\makeatletter
\renewcommand{\cleardoublepage}{
\clearpage\ifodd\c@page\else
\hbox{}
\vspace*{\fill}
\thispagestyle{empty}
\newpage
\fi}

%----------------------------------------------------------------------------------------
%	COMANDI PERSONALIZZATI
%----------------------------------------------------------------------------------------

\newcommand{\Tau}{\mathcal{T}}
\DeclareMathOperator{\sgn}{sgn}
\newcommand{\degree}{^\circ\text{C}} % SIMBOLO GRADI
\newcommand{\notimplies}{\mathrel{{\ooalign{\hidewidth$\not\phantom{=}$\hidewidth\cr$\implies$}}}}
\renewcommand{\qed}{\tag*{$\blacksquare$}}
\usepackage{xspace}
\newcommand{\latex}{\LaTeX\xspace}
\newcommand{\tex}{\TeX\xspace}
\newcommand{\questeq}{\overset{?}{=}} % è vero che?
\newcommand{\complementary}{^{\mathrm{C}}}
\newcommand{\transpose}{^{\mathrm{T}}}
\renewcommand{\emptyset}{\varnothing} % simbolo insieme vuoto
\newcommand{\Bot}{\perp \!\!\! \perp} % indipendenza
\renewcommand{\tilde}{\widetilde}
\renewcommand{\hat}{\widehat}
\renewcommand{\theta}{\vartheta}
\newcommand{\boxedText}[1]{\noindent\fbox{\parbox{\textwidth-0.5cm}{#1}}}

\usepackage{mathtools} % Serve per i comandi dopo
\DeclarePairedDelimiter{\abs}{\left\lvert}{\right\rvert}
\DeclarePairedDelimiter{\norm}{\left\lVert}{\right\rVert}
\DeclarePairedDelimiter{\sca}{\left\langle}{\right\rangle}

\newcommand{\fou}[1]{\mathcal{F}\left\{#1\right\}}
\newcommand{\ifou}[1]{\mathcal{F}^{-1}\left\{#1\right\}}
\newcommand{\Mc}{\mathcal{M}}
\newcommand{\E}{\mathbb{E}}
\newcommand{\R}{\mathbb{R}}
\newcommand{\N}{\mathbb{N}}
\newcommand{\C}{\mathbb{C}}
\newcommand{\Z}{\mathbb{Z}}
%\DeclarePairedDelimiter{\fou}{\mathcal{F}}{}
%\DeclarePairedDelimiter{\ifou}{\mathcal{F}^{-1}\left\{}{\right\}}

%----------------------------------------------------------------------------------------
%	APPENDICE
%----------------------------------------------------------------------------------------

\usepackage[toc,page]{appendix}

%----------------------------------------------------------------------------------------
%	SIMBOLI CARTE DA GIOCO
%
% Sono i seguenti:
%   \varheartsuit
%   \vardiamondsuit
%   \clubsuit
%   \spadesuit
%----------------------------------------------------------------------------------------

\DeclareSymbolFont{extraup}{U}{zavm}{m}{n}
\DeclareMathSymbol{\varheartsuit}{\mathalpha}{extraup}{86}
\DeclareMathSymbol{\vardiamondsuit}{\mathalpha}{extraup}{87}

%----------------------------------------------------------------------------------------

\definecolor{grey245}{RGB}{245,245,245}

\newtheoremstyle{blacknumbox} % Theorem style name
{0pt}% Space above
{0pt}% Space below
{\normalfont}% Body font
{}% Indent amount
{\bf\scshape}% Theorem head font --- {\small\bf}
{.\;}% Punctuation after theorem head
{0.25em}% Space after theorem head
{\small\thmname{#1}\nobreakspace\thmnumber{\@ifnotempty{#1}{}\@upn{#2}}% Theorem text (e.g. Theorem 2.1)
%{\small\thmname{#1}% Theorem text (e.g. Theorem)
\thmnote{\nobreakspace\the\thm@notefont\normalfont\bfseries---\nobreakspace#3}}% Optional theorem note

% Per gli unnumbered tolgo il \nobreakspace subito dopo {\small\thmname{#1} perché altrimenti c'è uno spazio tra Teorema e il ".", lo spazio lo voglio solo se sono numerati per distanziare Teorema e "(2.1)"
\newtheoremstyle{unnumbered} % Theorem style name
{0pt}% Space above
{0pt}% Space below
{\normalfont}% Body font
{}% Indent amount
{\bf\scshape}% Theorem head font --- {\small\bf}
{.\;}% Punctuation after theorem head
{0.25em}% Space after theorem head
{\small\thmname{#1}\thmnumber{\@ifnotempty{#1}{}\@upn{#2}}% Theorem text (e.g. Theorem 2.1)
%{\small\thmname{#1}% Theorem text (e.g. Theorem)
\thmnote{\nobreakspace\the\thm@notefont\normalfont\bfseries---\nobreakspace#3}}% Optional theorem note

\newcounter{dummy} 
\numberwithin{dummy}{chapter}

\theoremstyle{blacknumbox}
\newtheorem{definitionT}[dummy]{Definition}
\newtheorem{theoremT}[dummy]{Theorem}

\theoremstyle{unnumbered}
\newtheorem*{ossT}{Observation}
\newtheorem*{exT}{Example}
\newtheorem*{dimT}{Proof}

\RequirePackage[framemethod=default]{mdframed} % Required for creating the theorem, definition, exercise and corollary boxes

% orange box
\newmdenv[skipabove=7pt,
skipbelow=7pt,
rightline=false,
leftline=true,
topline=false,
bottomline=false,
linecolor=orange,
backgroundcolor=orange!5,
innerleftmargin=5pt,
innerrightmargin=5pt,
innertopmargin=5pt,
leftmargin=0cm,
rightmargin=0cm,
linewidth=2pt,
innerbottommargin=5pt]{oBox}

% green box
\newmdenv[skipabove=7pt,
skipbelow=7pt,
rightline=false,
leftline=true,
topline=false,
bottomline=false,
linecolor=green,
backgroundcolor=green!0,
innerleftmargin=5pt,
innerrightmargin=5pt,
innertopmargin=5pt,
leftmargin=0cm,
rightmargin=0cm,
linewidth=2pt,
innerbottommargin=5pt]{gBox}

% blue box
\newmdenv[skipabove=7pt,
skipbelow=7pt,
rightline=false,
leftline=true,
topline=false,
bottomline=false,
linecolor=blue,
backgroundcolor=blue!0,
innerleftmargin=5pt,
innerrightmargin=5pt,
innertopmargin=5pt,
leftmargin=0cm,
rightmargin=0cm,
linewidth=2pt,
innerbottommargin=5pt]{bBox}

% dim box
\newmdenv[skipabove=7pt,
skipbelow=7pt,
rightline=false,
leftline=true,
topline=false,
bottomline=false,
linecolor=black,
backgroundcolor=grey245!0,
innerleftmargin=5pt,
innerrightmargin=5pt,
innertopmargin=5pt,
leftmargin=0cm,
rightmargin=0cm,
linewidth=2pt,
innerbottommargin=5pt]{dimBox}

\newenvironment{theorem}{\begin{gBox}\begin{theoremT}}{\end{theoremT}\end{gBox}}	
\newenvironment{definition}{\begin{bBox}\begin{definitionT}}{\end{definitionT}\end{bBox}}	
\newenvironment{obs}{\begin{dimBox}\begin{ossT}}{\end{ossT}\end{dimBox}}
\newenvironment{example}{\begin{dimBox}\begin{exT}}{\end{exT}\end{dimBox}}
\newenvironment{dimostrazione}{\begin{dimBox}\begin{dimT}}{\[\qed\]\end{dimT}\end{dimBox}}

%----------------------------------------------------------------------------------------
%	INDICE INIZIALE
%----------------------------------------------------------------------------------------

\setcounter{secnumdepth}{3} % DI DEFAULT LE SUBSUBSECTION NON SONO NUMERATE, COSÌ SÌ
\setcounter{tocdepth}{2} % FISSA LA PROFONDITÀ DELLE COSE MOSTRATE NELL'INDICE

%\usepackage[hidelinks,bookmarksnumbered]{hyperref} % Rende l'indice interattivo e hidelinks nasconde il bordo rosso dai riferimenti

\usepackage{bookmark}% loads hyperref too
    \hypersetup{
        %pdftitle={Fundamentos de C\'alculo},
        %pdfsubject={C\'alculo diferencial},
        bookmarksnumbered=true,
        bookmarksopen=true,
        bookmarksopenlevel=1,
        hidelinks,% remove border and color
        pdfstartview=Fit, % Fits the page to the window.
        pdfpagemode=UseOutlines, %Determines how the file is opening in Acrobat; the possibilities are UseNone, UseThumbs (show thumbnails), UseOutlines (show bookmarks), FullScreen, UseOC (PDF 1.5), and UseAttachments (PDF 1.6). If no mode if explicitly chosen, but the bookmarks option is set, UseOutlines is used.
    }

\usepackage{glossaries} % certain packages that must be loaded before glossaries, if they are required: hyperref, babel, polyglossia, inputenc and fontenc
\setacronymstyle{long-short}

% Questo comando e quello dopo servono per avere il comando \tocless da mettere prima di una sezione che non voglio far apparire nell'indice
\newcommand{\nocontentsline}[3]{}
\newcommand{\tocless}[2]{\bgroup\let\addcontentsline=\nocontentsline#1{#2}\egroup}

%\usepackage{showframe}
\usepackage[textsize=tiny, textwidth=1.5cm]{todonotes} %add disable to options to not show in pdf

%\usepackage[nomarginpar,margin=1in]{geometry}

\usepackage[paperwidth=210mm,
            paperheight=297mm,
            left=1in,
            %top=50pt,
            textwidth=400pt,
            marginparsep=25pt,
            marginparwidth=1in,
            %textheight=692pt,
            footskip=50pt
            ]
           {geometry}


% Comandi pratici


\newcommand{\RR}{\mathbb{R}}

% d nell'integrale e i rispettivi usi
\newcommand{\de}{\,\mathrm d}
\newcommand{\dx}{\de x}
\newcommand{\dy}{\de y}
\newcommand{\dl}{\de l}
\newcommand{\dr}{\de r}
\newcommand{\ds}{\de s}
\newcommand{\dt}{\de t}
\newcommand{\dv}{\de v}
\newcommand{\dxi}{\de \xi}
\newcommand{\drho}{\de \rho}

% d nell'integrale con differenziale vettoriale
\newcommand{\dxx}{\de \x}
\newcommand{\dyy}{\de \y}
\newcommand{\dsig}{\de \sigg}

\allowdisplaybreaks[4] % Consente di rompere equazioni su più pagine

\newacronym{sp}{SP}{stochastic process}
\newacronym{ssp}{SSP}{stationary stochastic process}
\newacronym{ma}{MA}{Moving Average}
\newacronym{ar}{AR}{Auto Regressive}
\newacronym{arma}{ARMA}{Auto Regressive Moving Average}
\newacronym{mse}{MSE}{mean square error}
\newacronym{wn}{WN}{White Noise}
\newacronym{pem}{PEM}{Prediction Error Minimization}
\DeclareMathOperator{\WN}{WN}
\DeclareMathOperator*{\argmax}{arg\,max}
\DeclareMathOperator*{\argmin}{arg\,min}

% \usepackage{xifthen}
% 1 parametro necessario, con valore "nullo" di default (le due quadre vuote)
% non mettere le seconde quadre equivale a dire che è obbligatorio
% \newcommand{\ar}[1][]{\ifthenelse{\isempty{#1}}{AR}{AR($#1$)}}
% \newcommand{\ma}[1][]{\ifthenelse{\isempty{#1}}{MA}{MA($#1$)}}
% \newcommand{\arma}[1][]{\ifthenelse{\isempty{#1}}{ARMA}{ARMA($#1$)}}
% \newcommand{\arx}[1][]{\ifthenelse{\isempty{#1}}{ARX}{ARX($#1$)}}

%%%%%%%%%%%%%%%%%%%%%%%%%%%%%%%%%%%%%%%%%%%%%%%
%%%%%%%%%%%%%%%%%%%%%%%%%%%%%%%%%%%%%%%%%%%%%%%

\begin{document}

%%%%%%%%%%%%%%%%%%%%%%%%%%%%%%%%%%%%%%%%%%%%%%%
%%%%%%%%%%%%%%%%%%%%%%%%%%%%%%%%%%%%%%%%%%%%%%%

\frontmatter
\pagestyle{empty}
\vspace*{\fill}
\begin{center}
	{\large \textsc{Lecture Notes of}}\\
	\vspace*{0.4cm}
	{\Huge \textsc{Model Identification}}\\
	\vspace*{0.4cm}
	{\Huge \textsc{and Data Analysis}}\\
	\vspace*{1cm}
	{\large {From Professor Simone Garatti's lectures}}\\
	\vspace*{0.1cm}
	{\large for the MSc in Mathematical Engineering}\\
	\vspace*{0.4cm}
	{\large {by Teo Bucci, Filippo Cipriani \& Gabriele Corbo}}\\
	\vspace*{1cm}
	Politecnico di Milano\\A.Y. 2021/2022
\end{center}
\vspace*{\fill}
\newpage

%%%%%%%%%%%%%%%%%%%%%%%%%%%%%%%%%%%%%%%%%%%%%%%
%%%%%%%%%%%%%%%%%%%%%%%%%%%%%%%%%%%%%%%%%%%%%%%

{\Large \textit{Lecture Notes of Model Identification and Data Analysis}}

\vspace*{\fill}

%\textcopyright \ Authors.

This text is provided under Creative Commons BY-NC-SA 4.0 license.\\
\url{https://creativecommons.org/licenses/by-nc-sa/4.0/}

The \LaTeX \ source code is available at\\
\url{https://github.com/teobucci/mida}

\vspace*{1cm}

Revision of \today

Developed by\\
Teo Bucci - \texttt{teo.bucci@mail.polimi.it}\\
Filippo Cipriani - \texttt{filippo.cipriani@mail.polimi.it}\\
Gabriele Corbo - \texttt{gabriele.corbo@mail.polimi.it}\\ \\
Compiled with \ensuremath\heartsuit \\

%\textbf{Prefazione}

Please notify errors or changes through an email or a pull request.

\newpage

%%%%%%%%%%%%%%%%%%%%%%%%%%%%%%%%%%%%%%%%%%%%%%%
%%%%%%%%%%%%%%%%%%%%%%%%%%%%%%%%%%%%%%%%%%%%%%%

% INDICE
\addtocontents{toc}{\protect\thispagestyle{empty}}
\tableofcontents
%\newpage

%%%%%%%%%%%%%%%%%%%%%%%%%%%%%%%%%%%%%%%%%%%%%%%
%%%%%%%%%%%%%%%%%%%%%%%%%%%%%%%%%%%%%%%%%%%%%%%

% PAGINA VUOTA PER FAR PARTIRE IL CAPITOLO IN UNA PAGINA DISPARI
%\myNewEmptyPage

\AtEndDocument{\cleardoublepage}

%%%%%%%%%%%%%%%%%%%%%%%%%%%%%%%%%%%%%%%%%%%%%%%
%%%%%%%%%%%%%%%%%%%%%%%%%%%%%%%%%%%%%%%%%%%%%%%

\mainmatter
\pagestyle{fancy} % Riswitcha per riavere il numero pagina
%\setcounter{page}{1} % Fa ripartire il contatore pagina da 1

%%%%%%%%%%%%%%%%%%%%%%%%%%%%%%%%%%%%%%%%%%%%%%%
%%%%%%%%%%%%%%%%%%%%%%%%%%%%%%%%%%%%%%%%%%%%%%%

\tikzstyle{block}      = [draw, rectangle, inner sep=6pt]
\tikzstyle{every node} = [font=\small]
\tikzstyle{sum}        = [draw, circle, inner sep=6pt]
%\tikzstyle{input} = [coordinate]
%\tikzstyle{output} = [coordinate]
%\tikzstyle{pinstyle} = [pin edge={to-,thin,black}]

\part{MIDA I}

%!TEX root = ../main.tex

%!TEX root = ../main.tex
\chapter{Stochastic Processes and Model Classes}
A \gls{sp} is an infinite sequence of random variables all defined on the same probabilistic space $(\Omega,\mathcal{A},\mathbb{P})$:
\[
	\ldots,v(1,s),v(2,s),v(3,s),\ldots,v(t,s),\ldots
\]
with:
\begin{itemize}
 	\item $s$: random experiment realization;
 	\item $t$: time index.
 \end{itemize}

\begin{obs}
	\gls{sp} extends the notion of random vector (\gls{sp} is a random vector with infinite entries).
\end{obs}

\begin{obs}
For a \emph{fixed} value of the random experiment $s = \overline{s}$, the \gls{sp} becomes the numeric sequence:
\[
	\ldots,v(1,\overline{s}),v(2,\overline{s}),v(3,\overline{s}),\ldots,v(t,\overline{s}),\ldots
\]
which is called \textbf{realization} of the \gls{sp}.
For different values of $s$, one gets different realizations of the \gls{sp}.
\end{obs}

We will think of available observations ${u(1),u(2),\ldots,u(N)}$ and ${y(1), y(2),\ldots, y(N)}$ as \emph{finite} length realizations.

\begin{definition}[Mean value]
	The mean value $m(t)$ is the expected value of the random variable $v(t,s)$ at time $t$:
	\[
		m(t)=\E[v(t, s)]=\int_{\Omega} v(t, s) \mathbb{P}(ds)
	\]
	$m(t)$ returns the value around which the process take value at time $t$.
\end{definition}

\begin{definition}[Covariance function]
	The covariance function $\gamma(t_{1}, t_{2})$ is the expected value of the product of unbiased random variables $(v(t, s)-m(t))$ at two time instants $(t_{1}, t_{2})$:
	\begin{align*}
		\gamma(t_{1}, t_{2}) &=\E[(v(t_{1}, s)-m(t_{1}))(v(t_{2}, s)-m(t_{2}))] \\
		&=\int_{\Omega}(v(t_{1}, s)-m(t_{1}))(v(t_{2}, s)-m(t_{2})) \mathbb{P}(ds)
	\end{align*}
	$\gamma(t_{1}, t_{2})$ quantifies the relation existing between the process realizations and the mean value at two different time instants.
\end{definition}

Particular case: $t_{1}=t_{2}=t$.
\begin{definition}[Variance]
	The variance quantifies the process dispersion around its mean value at each time instant:
	\[
		\gamma(t, t)=\E[(v(t, s)-m(t))^{2}]=\int_{\Omega}(v(t, s)-m(t))^{2} \mathbb{P}(ds)
	\]
\end{definition}

\section{Stationary Stochastic Processes}
\begin{definition}
	A stochastic process is called \textbf{\gls{ssp}} (wide-sense) if:
	\begin{itemize}
		\item $m(t)=m, \forall t$;
		\item $\gamma(t_{1}, t_{2})$ only depends on $\tau=t_{1}-t_{2}$,\\
		i.e. $\gamma(t_{1}, t_{2})=\gamma(t_{3}, t_{4})$ if $t_{1}-t_{2}=t_{3}-t_{4}=\tau, \forall t_{1}, t_{2}, t_{3}, t_{4}$.
	\end{itemize}
\end{definition}
The idea is that the probabilistic properties of a \gls{ssp} are time-translation invariant.

\glspl{ssp} admit a \emph{simplified} representation of the covariance function:
\[
	\boxed{\gamma(\tau)=\gamma(t, t-\tau)=\E[(v(t)-m)(v(t-\tau)-m)]}
\]
where
\[
	\boxed{\gamma(0)=\E[(v(t)-m)^{2}]=\lambda^2} \quad \text{is the variance of the process}
\]
Why \glspl{ssp}?
\begin{itemize}
	\item \emph{Stationary} means \emph{time-invariant} data generating system (situation often encountered in practice).
	\item \glspl{ssp} are easier to study.
	\item Non-stationary processes can be recast in the framework of \gls{ssp} by first eliminating the non-stationary part from data (data pre-processing).
\end{itemize}

\textbf{Properties of the covariance function for a \gls{ssp}.}
\begin{itemize}
	\item $\gamma(0)=\E[(v(t)-m)^{2}] \geq 0$ (non negative at initial time).
	\item $|\gamma(\tau)| \leq \gamma(0)$ (bounded), indeed consider the non-negative quantity:
	\begin{align*}
		\E[(v(t) \pm v(t-\tau))^2] &= \E[v(t)^2] \pm \E[2v(t-\tau)v(t)]+\E[v(t-\tau)^2]\\
		&=\E[v(t)^2] \pm 2\gamma(\tau)+\E[v(t)^2]\\
		&=2\gamma(0) \pm 2\gamma(\tau) \geq 0 
		\iff -\gamma(0) \leq \gamma(\tau) \leq \gamma(0) \iff |\gamma(\tau)| \leq \gamma(0)
	\end{align*}
	\item $\gamma(\tau)=\gamma(-\tau)$ (symmetric), indeed:
	\begin{align*}
		\gamma(-\tau)&=\E[(v(t)-m)(v(t-(-\tau))-m)]\\
		&=\E[(v(t)-m)(v(t+\tau))-m)]\\
		&=\E[(v(t+\tau)-m)(v(t))-m)]\\
		&=\gamma(\tau) \quad(t+\tau-t=\tau)
	\end{align*}
\end{itemize}

\fg{0.7}{Screen Shot 2022-03-06 at 00.50.16}

Given a \gls{ssp} $x(t)$, we will write $m_{x}$ e $\gamma_{x}(\tau)$ for its mean and covariance function

Two \glspl{ssp} $y_{1}(t)$ and $y_{2}(t)$ are wide-sense equivalent if $m_{y_{1}}=m_{y_{2}}$ e $\gamma_{y_{1}}(\tau)=\gamma_{y_{2}}(\tau), \forall \tau$

The \emph{covariance function}
$$
	\E[(v(t)-m) \cdot(v(t-\tau)-m)]
$$
is very \emph{different} from the $2^{\text{nd}}$ order moment function $\E[v(t) \cdot v(t-\tau)]$.

\begin{example}
$\boxed{v(t,s)=\alpha (s)}$, where $\alpha (s)\sim \mathcal{N}(1,3)$.

\begin{itemize}
	\item $m_{v}(t)=\E[v(t, s)]=\E[\alpha(s)]=1=m_{v}$\\
	doesn't depend on $t$;
	\item $\begin{aligned}[t]
		\gamma_{v}(t, t-\tau)&=\E[(v(t, s)-m_{v}(t))(v(t-\tau, s)-m_{v}(t-\tau))]\\
		&=\E[(\alpha(s)-1)(\alpha(s)-1)]=3=\gamma_{v}(\tau)
	\end{aligned}$\\
	doesn't depend on $t$.\\
	Then the process is stationary.
\end{itemize}
\end{example}

\begin{example}
$\boxed{v(t, s)=t \cdot \alpha(s)-t}$, where $\alpha(s) \sim \mathcal{N}(1,3)$.
\begin{itemize}
	\item $m_{v}(t)=\E[v(t, s)]=\E[t \cdot \alpha(s)-t]=t \cdot \E[\alpha(s)]-t=t-t=0$\\
	doesn't depend on $t$;
	\item $\begin{aligned}[t]
		\gamma_{v}(t, t-\tau)&=\E[(v(t, s)-m_{v}(t))(v(t-\tau, s)-m_{v}(t-\tau))]\\
	&=\E[(t \cdot \alpha(s)-t)((t-\tau) \cdot \alpha(s)-(t-\tau))]\\
	&=\E[t(t-\tau)(\alpha(s)-1)^{2}]\\
	&=t(t-\tau) \cdot \E[(\alpha(s)-1)^{2}]=t \cdot(t-\tau) \cdot 4
	\end{aligned}$\\
	\emph{does} depend on $t$.\\
	Then the process is not stationary.
\end{itemize}
\end{example}

\begin{obs}
If $\gamma(t, \tau)>0$ then there is a tendency of preserving the sign going from $t$ to $\tau $. The opposite otherwise.
\end{obs}
\section{White Noise}

\begin{definition}
	An \gls{ssp} $e(t)$ is called \textbf{\gls{wn}} with mean $\mu$ and variance $\lambda^{2}$, we shall write
	\[
		\boxed{e(t) \sim \WN(\mu, \lambda^{2})}
	\]
	if the following conditions hold:
	\begin{itemize}
		\item $\E[e(t)]=\mu \quad \forall t$
		\item $\gamma_{e}(0)=\E[(e(t)-\mu)^{2}]=\lambda^{2} \quad \forall t$
		\item $\gamma_{e}(\tau)=\E[(e(t)-\mu) \cdot(e(t-\tau)-\mu)]=0 \quad \forall t, \forall \tau \neq 0$
	\end{itemize}	
\end{definition}
The last property is the fundamental one. It says that there is complete incorrelation between random variables at different time instants. The realizations of $e(t)$ are erratic and unpredictable (\textbf{whiteness property}).

%\fg{0.7}{Screen Shot 2022-03-08 at 09.53.35}
\begin{figure}[htpb]
	\centering
	\begin{tikzpicture}
		\draw [-stealth] (-4,0) -- (4,0) node [at end,below] {$\tau$};
		\foreach \x in {-3,-2,-1,1,2,3}{
		    \node [circle,inner sep=1.5pt,fill=black,label=below:{$\x$}] at (\x,0) {};
		}
		\node [below] at (0,0) {$0$};
		\draw [-stealth] (0,0) -- (0,3) node [at end,right] {$\gamma_e(\tau)$};
		\node [circle,inner sep=1.5pt,fill=black,label=right:{$\lambda^2$}] at (0,2) {};
	\end{tikzpicture}
\end{figure}
\FloatBarrier

\begin{obs}
The probability distribution of each single random variables $e(t,s)$ does not matter and is not made explicit in general (wide-sense description of \gls{ssp}).
It could be Gaussian, uniform, etc. (WGN = White Gaussian Noise, WUN = White Uniform Noise, etc.).
\end{obs}

\begin{obs}
Is a constant realization admissible? Yes, it is, but such realization is \emph{highly unlikely}.
\end{obs}
\gls{wn} is a sort of \emph{building block} to construct a number of different \glspl{ssp}.

To ease the notation, in the following we will consider \emph{zero mean} \glspl{wn}. The extension to the general case presents no conceptual difficulties.

\subsection{MA processes}

\begin{definition}
	Let $e(t) \sim \WN(0, \lambda^{2})$. A \textbf{\gls{ma}} process of order $n$ is obtained as:
	\[
		\boxed{y(t)=c_{0} e(t)+c_{1} e(t-1)+c_{2} e(t-2)+\cdots+c_{n} e(t-n)}
	\]
\end{definition}
In other words, the output $y(t)$ of a MA process is given by a linear combination of the last $n+1$ past values of the input \gls{wn} $e(t)$.
While $t$ is let vary, the linear combination is made on a sliding window (moving average).

\textbf{Mean.}
\[
	m_{y}(t)=\E[y(t)] = \E[c_{0} e(t)+c_{1} e(t-1)+c_{2} e(t-2)+\cdots+c_{n} e(t-n)] = 0+\cdots+0=m_{y}=0,
\]
hence $m_{y}(t)$ doesn't depend on $t$.


%!TEX root = ../main.tex
\textbf{Variance.} (i.e. covariance when $\tau =0$)
\begin{align*}
	\gamma (0)&=\E[(y(t)-m_{y})(y(t)-m_{y})]=\E[(y(t))^2]\\
	&=\E[(c_{0} e(t)+c_{1} e(t-1)+\ldots+c_{n} e(t-n))^{2}]\\
	&=\E[c_{0}^{2} e(t)^{2}+\ldots+c_{n}^{2} e(t-n)^{2}\\
	&\qquad+2 c_{0} c_{1} e(t) e(t-1)+\ldots+2 c_{n-1} c_{n} e(t-n-1) e(t-n)]\\
	&=c_{0}^{2} \E[e(t)^{2}]+c_{1}^{2} \E[e(t-1)^{2}]+\ldots+c_{n}^{2} \E[e(t-n)^{2}]\\
	&\qquad+2 c_{0} c_{1} \E[e(t) e(t-1)]+\ldots+2 c_{n-1} c_{n} \E[e(t-n-1) e(t-n)]
\end{align*}

Since $e(t) \sim \WN(0, \lambda^{2})$, we have that:
$$
\E[e(t)^{2}]=\E[e(t-1)^{2}]=\ldots=\E[e(t-n)^{2}]=\lambda^{2}
$$
and that
$$
\E[e(t) e(t-1)]=\ldots=\E[e(t-n-1) e(t-n)]=0
$$
thus
\[
	\boxed{\gamma (t,0)=\gamma (0)=(c_{1}^2 +c_{1}^2 +\cdots+c_{n}^2 )\cdot\lambda^2}
\]
hence $\gamma (0)$ doesn't depend on $t$.

\textbf{Covariance.}

To calculate the generic covariance, let us proceed with $\tau =1$.
\begin{align*}
	\gamma(t, t-1)&=\E[(y(t)-m_{y})(y(t-1)-m_{y})]\\
	&=\E[y(t) y(t-1)]\\
	&=\E[(c_{0} e(t)+c_{1} e(t-1)+\cdots+c_{n} e(t-n))\cdot (y(t-1))]\\
	&=\E[(c_{0} e(t)+c_{1} e(t-1)+\cdots+c_{n} e(t-n))\cdot (c_{0} e(t-1)+\cdots+c_{n-1} e(t-n)+c_{n} e(t-n-1))]
\end{align*}

Only those terms where the \gls{wn} is multiplied by itself at the same time instant are non null.
\[
	\gamma(t, t-1)=\gamma (1)=(c_{0}c_{1}+c_{1}c_{2}+\cdots+c_{n-1}c_{n})\cdot\lambda^2
\]
hence $\gamma (1)$ doesn't depend on $t$.

Similarly
\begin{align*}
	\gamma (t,t-2)=\gamma (2) &= (c_{0}c_{2}+c_{1}c_{3}+\cdots+c_{n-2}c_{n})\cdot\lambda^2\\
	&\vdots\\
	\gamma (t,t-n)=\gamma (n) &= (c_{0}c_{n})\cdot\lambda^2\\
	\gamma (t,t-n-1)=\gamma (n+1) &= 0
\end{align*}
since all products are uncorrelated. In conclusion
\[
	\gamma (\tau )=\begin{cases}
		(c_{1}^2 +c_{1}^2 +\cdots+c_{n}^2 )\cdot\lambda^2 & \text{if}\ \tau =0\\
		(c_{0}c_{1}+c_{1}c_{2}+\cdots+c_{n-1}c_{n})\cdot\lambda^2 & \text{if}\ \tau =\pm 1\\
		(c_{0}c_{2}+c_{1}c_{3}+\cdots+c_{n-2}c_{n})\cdot\lambda^2 & \text{if}\ \tau =\pm 2\\
		\vdots\\
		(c_{0}c_{n})\cdot\lambda^2 & \text{if}\ \tau =\pm n\\
		0 & \text{if}\ |\tau| > \pm n\\
	\end{cases}
\]
\subsection{MA(\texorpdfstring{$\infty$}{infinity}) processes}

\[
	y(t)=c_{0} e(t)+c_{1} e(t-1)+\cdots+c_{i} e(t-i)+\cdots=\sum_{i=0}^{\infty} c_{i} e(t-i) \quad e(t) \sim \WN\left(0, \lambda^{2}\right)
\]
Assumption: $\sum_{i=0}^{\infty} c_{i}^{2}<\infty$ (it guarantees that $y(t)$ is well defined).

\textbf{Mean.}
\[
	m_{y}(t)=\E[y(t)]=\E\left[\sum_{i=0}^{\infty} c_{i} e(t-i)\right]=\sum_{i=0}^{\infty} c_{i} \E[e(t-i)]=\sum_{i=0}^{\infty} c_{i} \cdot 0=0
\]
doesn't depend on $t$.

\textbf{Variance.}
\begin{align*}
	\gamma_{y}(t, t)&=\E[(y(t)-m_{y})^{2}]\\
	&=\E\left[\sum_{i=0}^{\infty} c_{i} e(t-i) \cdot \sum_{j=0}^{\infty} c_{j} e(t-j)\right]\\
	&=\E\left[\sum_{i, j=0}^{\infty} c_{i} c_{j} \cdot e(t-i) e(t-j)\right]\\
	&=\sum_{i, j=0}^{\infty} c_{i} c_{j} \cdot \E[e(t-i) e(t-j)]\\
	&=\{\text{non null only when }i=j\}\\
	&=\sum_{i=0}^{\infty} c_{i}^2 \cdot\lambda^2 
\end{align*}
doesn't depend on $t$.

\textbf{Covariance.}
\begin{align*}
	\gamma_{y}(t, t-\tau) &=\E[(y(t)-m_{y}) (y(t-\tau)-m_{y})]\\
	&=\E[y(t) y(t-\tau)]\\
	&=\E\left[\sum_{i=0}^{\infty} c_{i} e(t-i) \cdot \sum_{j=0}^{\infty} c_{j} e(t-j-\tau)\right]\\
	&=\E\left[\sum_{i, j=0}^{\infty} c_{i} c_{j} \cdot e(t-i) e(t-j-\tau)\right]\\
	&=\sum_{i, j=0}^{\infty} c_{i} c_{j} \cdot \E[e(t-i) e(t-j-\tau)]\\
	&=\{\text{non null only when }i=j+\tau\}\\
	&=\sum_{j=0}^{\infty} c_{j+\tau}c_{j}\cdot\lambda^2 
\end{align*}
doesn't depend on $t$.

So if $\sum_{i=0}^{\infty} c_{i}^{2}<\infty$ then the MA($\infty $) process is well defined and is a \gls{ssp}.

\textbf{Observation.} MA($\infty$) processes are very general, they almost \emph{cover} the class of \glspl{ssp} (i.e. apart from few exceptions, all \gls{ssp} can be written as MA($\infty$)).

However, MA($\infty$) are difficult to handle since there are infinite coefficients and, moreover, the computation of the covariance function requires the computation of the sum of an infinite series (hard in general).

On the other hand, MA($n$) are too limited, that is why we will look into AR and ARMA models.
%!TEX root = ../main.tex

%!TEX root = ../main.tex
\section{AR and ARMA processes}

\subsection{AR processes}

\begin{definition}
	Let $e(t) \sim \WN(0, \lambda^{2})$. An \textbf{\gls{ar}} process of order $m$ is obtained as:
	\[
		\boxed{y(t)=a_{1} y(t-1)+a_{2} y(t-2)+\cdots+a_{m} y(t-m)+e(t)}
	\]
\end{definition}

Terminology:
\begin{itemize}
	\item $a_{1}, a_{2}, \ldots, a_{m}$: model coefficients.
	\item $m$: process (model) order.
\end{itemize}
 
Hence, the output $y(t)$ of an AR process is recursively defined as the linear combination of last $m$ past values of the process itself plus the input $e(t)$ at the same time instant.

\begin{obs}
The difference equation generating the AR process admits non-unique solution unless we specify an \emph{initial condition}. Which solution do we consider as the AR process?

By AR process we mean the solution obtained by taking the initial condition $\boxed{y(t_{0})=0}$ and letting $t_{0} \to -\infty$ (in short, we will write $y(-\infty)=0$)). In other words, the AR process is the \textbf{steady-state solution}.
\end{obs}

\begin{example}
Let us consider the AR($1$) process defined by $y(t)=a y(t-1)+e(t)$, where $e(t) \sim \WN\left(\mu, \lambda^{2}\right)$

What is the steady state solution?
\begin{align*}
	y(t) & =a y(t-1)+e(t) & (y(t-1)=a y(t-2)+e(t-1)) \\
	& =e(t)+a e(t-1)+a^{2} y(t-2) & (y(t-2)=a y(t-3)+e(t-2)) \\
	& \vdots & \\
	& =e(t)+a e(t-1)+a^{2} e(t-2)+\cdots+a^{t-t_{0}} y\left(t_{0}\right) & (y\left(t_{0}\right)=0,t_{0} \to-\infty) \\
	& \vdots & \\
	& =e(t)+a e(t-1)+a^{2} e(t-2)+\cdots+a^{n} e(t-n)+\cdots\\
	&=\sum_{i=0}^{\infty} a^{i} e(t-i)
\end{align*}
The steady state solution is an MA($\infty$) process with coefficients: $c_{0}=1, c_{1}=a, c_{2}=a^{2}, \ldots, c_{i}=a^{i}, \ldots$

In general, AR processes are MA($\infty$) processes with coefficients determined by the AR model coefficients by recursively apply the difference equation.

MA($\infty$) processes are well defined if:
\begin{align*}
	\sum_{i=0}^{\infty} \left(c_{i}\right)^2=\sum_{i=0}^{\infty} \left(a^{i}\right)^2=\sum_{i=0}^{\infty} \left(a^{2}\right)^i< +\infty \iff a^2< 1
\end{align*}

So if $|a|<1$ the steady-state solution is well defined and \gls{ssp}.
\end{example}

\subsection{ARMA processes}

\begin{definition}
	Let $e(t) \sim \WN(\mu, \lambda^{2})$. An \textbf{\gls{arma}} process of order $m,n$ is obtained as:
	\[
		\boxed{
			\begin{aligned}
				y(t)&=a_{1} y(t-1)+a_{2} y(t-2)+\cdots+a_{m} y(t-m)\quad & \text{AR($m$) part}\\
				&\qquad+c_{0} e(t)+c_{1} e(t-1)+\cdots+c_{n} e(t-n) . \quad & \text{MA($n$) part}
			\end{aligned}
		}
	\]
\end{definition}

Again by ARMA process we mean the steady-state solution obtained by letting $y(-\infty)=0$.
 
Similarly to AR processes, the steady-state solution is an MA($\infty)$ process whose coefficients are obtained from the ARMA model coefficients by recursively apply the difference equation.
 
Terminology:
\begin{itemize}
	\item $m$ AR part order.
	\item $n$ MA part order.
\end{itemize}

An ARMA process is well defined and stationary under some conditions which are too complicated to verify directly.
We'll see how to solve this problem after introducing the operatorial representation of ARMA processes.

\subsection{Operatorial representation of ARMA processes}

\begin{definition}[Backward and forward shift operators]
	We define two useful operators as:
	\begin{itemize}
		\item Backward shift operator $z^{-1}$ is defined as: $z^{-1} x(t)=x(t-1)$.
		\item Forward shift operator $z$ is defined as: $z x(t)=x(t+1)$.	
	\end{itemize}
	
\end{definition}

\textbf{Properties of operators $z^{-1}$ and $z$.}

\begin{itemize}
	\item $z^{-1}$ and $z$ are \textbf{linear}:
		\begin{align*}
			z^{-1}(a \cdot x(t)+b \cdot y(t))&=a \cdot x(t-1)+b \cdot y(t-1) \\
			z(a \cdot x(t)+b \cdot y(t))&=a \cdot x(t+1)+b \cdot y(t+1)
		\end{align*}
	\item $z^{-1}$ and $z$ can be \textbf{recursively applied}:
		\[
			z^{-1}(z^{-1}(z^{-1}(x(t))))=z^{-1}(z^{-1}(x(t-1)))=z^{-1}(x(t-2))=x(t-3)=z^{-3} x(t)
		\]
		similarly for $z$.
	\item $z^{-1}$ and $z$ can be \textbf{linearly composed}:
		\begin{align*}
			(a z^{-1}+b z+c z^{-3}+d z^{2}) x(t)&=a(z^{-1} x(t))+b(z x(t))+c(z^{-3} x(t))+d(z^{2} x(t))= \\
			&=a x(t-1)+b x(t+2)+c x(t-3)+d x(t+2)
		\end{align*}
\end{itemize}

We can rewrite an ARMA process as follows:
\begin{align*}
	\left(1-a_{1} z^{-1}-a_{2} z^{-2}-\cdots-a_{m} z^{-m}\right) y(t)=\left(c_{0}+c_{1} z^{-1}+\cdots+c_{n} z^{-n}\right) e(t)
\end{align*}

Even more compact notation:
\[
	\boxed{y(t)=\frac{\left(c_{0}+c_{1} z^{-1}+\cdots+c_{n} z^{-n}\right)}{\left(1-a_{1} z^{-1}-a_{2} z^{-2}-\cdots-a_{m} z^{-m}\right)} e(t)=\frac{C(z)}{A(z)} e(t)}
\]
where:
\[
	\boxed{C(z)=\left(c_{0}+c_{1} z^{-1}+\cdots+c_{n} z^{-n}\right)} \quad \boxed{A(z)=\left(1-a_{1} z^{-1}-a_{2} z^{-2}-\cdots-a_{m} z^{-m}\right)}
\]
$\frac{C(z)}{A(z)}$ is called \textbf{discrete time transfer function} and it simply says that $y(t)$ is generated as the steady-state output of a linear operator that receive as input $e(t)$.
%!TEX root = ../main.tex
Say that $y(t)=W(z)u(t)$, where $W(z)=\frac{C(z)}{A(z)}$ means that $y(t)$ is the steady state solution to the recursive equation:
\[
	A(z)y(t)=C(z)u(t)
\]

\section{Composition of transfer functions and output processes}
\subsection{Series}
Given $u(t)$ stochastic process, consider
\begin{align*}
	x(t)&=W_{1}(z)u(t)=\frac{C_{1}(z)}{A_{1}(z)}u(t)\\
	y(t)&=W_{2}(z)x(t)=\frac{C_{2}(z)}{A_{2}(z)}x(t)
\end{align*}
\fg{0.7}{Screen Shot 2022-03-08 at 16.28.23}
\begin{theorem}
	The process $y(t)$ is the steady state output of a new filter having transfer function $W_{1}(z)\cdot W_{2}(z)$ fed by $u(t)$. That is,
	\[
		y(t)=[W_{1}(z)\cdot W_{2}(z)]u(t)=\frac{C_{1}(z)\cdot C_{2}(z)}{A_{1}(z)\cdot A_{2}(z)}u(t)
	\]
	meaning that $y(t)$ is the solution to the recursive equation $A_{1}(z)\cdot A_{2}(z) y(t) = C_{1}(z)\cdot C_{2}(z)u(t)$.
\end{theorem}

\subsection{Parallel}
Given $u(t)$ stochastic process, consider
\begin{align*}
	y_{1}(t)&=W_{1}(z)u(t)\\
	y_{2}(t)&=W_{2}(z)u(t)\\
	y(t)&=y_{1}(t)+y_{2}(t)=W_{1}(z)u(t)+W_{2}(z)u(t)=[W_{1}(z)+W_{2}(z)]u(t)
\end{align*}
\fg{0.7}{Screen Shot 2022-03-08 at 16.36.03}
\begin{theorem}
	The process $y(t)$ is the steady state output of a new filter having transfer function $W_{1}(z)+W_{2}(z)$ fed by $u(t)$.
\end{theorem}

\section{Poles and zeros}

\textbf{Remark.}
One can always multiply a transfer function by $z^{m}/z^{m}$.

Consider now $W(z)$ a complex-valued transfer function. Then one can identify:
\begin{itemize}
	\item \textbf{zeros}: all $z\in \mathbb{C}$ such that $W(z)=0$.
	\item \textbf{poles}: all $z\in \mathbb{C}$ such that $W^{-1} (z)=0$.
\end{itemize}
When $C(z),A(z)$ are polynomials with positive powers, then
\[
	\text{zeros}=\{z:(C(z)=0\} \qquad \text{poles}=\{z:(A(z)=0\}
\]

\textbf{Example.}
\begin{align*}
y(t) &=e(t)+\frac{1}{2} e(t-1)+\frac{1}{4} e(t-2) =\left(1+\frac{1}{2} z^{-1}+\frac{1}{4} z^{-2}\right) e(t) \\
&=\frac{1+\frac{1}{2} z^{-1}+\frac{1}{4} z^{-2}}{1} \cdot \frac{z^{2}}{z^{2}}\cdot e(t) =\frac{z^{2}+\frac{1}{2} z+\frac{1}{4}}{z^{2}} e(t)
\end{align*}
Poles are $z_{1}=z_{2}=0$.\\
Zeros are $z$ such that $z^{2}+\frac{1}{2} z+\frac{1}{4}=0$
\[
	z_{1,2}=\frac{-\frac{1}{2} \pm \sqrt{\left( \frac{1}{2}  \right) ^2 -4\cdot\frac{1}{4} } }{2} = -\frac{1}{4}\pm i\frac{\sqrt{3} }{4}
\]
\fg{0.7}{Screen Shot 2022-03-08 at 16.57.50}

\textbf{Remark.}
We say that $W(z)=C(z)/A(z)$ is
\begin{itemize}
	\item \textbf{asymptotically stable} if all \emph{poles} are such that $|z|<1$.
	\item \textbf{minimum phase} if all \emph{zeros} are such that $|z|<1$.
\end{itemize}

When ARMA processes are well-defined? Consider $W(z)$ rational transfer function, $v(t)$ stochastic process (input), $y(t)=W(z)v(t)$.

\begin{theorem}
	If
	\begin{itemize}
		\item $v(t)$ is a \emph{stationary} stochastic process;
		\item $W(z)$ is asymptotically stable;
	\end{itemize}
	then $y(t)$ is well-defined and is \emph{stationary}.
\end{theorem}

In the case of ARMA processes, the input is a White Noise, which is stationary by definition. Thus we just need to check the second condition.

In general one can factorize the denominator:
\begin{align*}
	y(t)&=\frac{C(z)}{A(z)}v(t)=\frac{C(z)}{(z-p_{1})(z-p_{2})\cdots(z-p_{m})}v(t)\\
	&=C(z)\cdot\frac{1}{z-p_{1}}\cdot\frac{1}{z-p_{2}}\cdots\frac{1}{z-p_{m}}v(t)\\
	&=\frac{1}{z-p_{m}}\left[ \frac{1}{z-p_{m-1}}\left[ \cdots\frac{1}{z-p_{1}}\left[ C(z)v(t) \right]   \right]   \right]  
\end{align*}
\fg{0.7}{Screen Shot 2022-03-08 at 17.10.36}
%!TEX root = ../main.tex
Now, let us consider the \gls{sp} $y(t)$ obtained as output of an asymptotically stable digital filter $F(z)$ fed by a \gls{ssp} $v(t)$ as input, but with a generic initialization (not steady-state output).

\fg{0.4}{Screenshot (17)}

\begin{theorem}
	There is just one stationary output which corresponds to the steady-state solution. However, if $F(z)$ is asymptotically stable, then all possible outputs obtained for different initialization of the digital filter $F(z)$ tends asymptotically (as $t \rightarrow \infty$) to the steady-state solution, i.e. to the stationary output.
\end{theorem}

\fg{0.7}{Screenshot (18)}

\section{Weak(Wide sense) characterization of AR,ARMA processes}
\textbf{Goal.} Given an AR, ARMA process compute the mean $m_y$ and covariance function $\gamma_y(\tau)$.

%Since the steady-state solution is an MA($\infty$) process we could use such results but are too diffucult

We'll rely on the recursive equation characterizing such processes.

\subsection{AR processes}
\textbf{Example.}

Let us consider the AR($1$) (or equivalently ARMA($1,0$)) process generated according to:
\begin{align*}
	y(t)=a \cdot y(t-1)+e(t) \quad \text{where} \quad e(t) \sim \WN\left(0, \lambda^{2}\right)
\end{align*}

\begin{itemize}
	\item Is $y(t)$ stationary?
	\item Compute $m_{y}$ and $\gamma_{y}(\tau)$ for $\tau=0, \pm 1, \pm 2, \ldots$
\end{itemize}

\todo{have a check here}
Operatorial representation for $y(t)$:

\begin{align*}
	y(t)&=z^{-1} a y(t)+e(t) \\
	\left(1-z^{-1} a\right) y(t)&=e(t) \\
	y(t)&=\frac{1}{1-z^{-1} a} e(t)
\end{align*}

The transfer function with positive powers (to identify zeroes and poles) is $y(t)=\frac{z}{z-a} e(t) \quad$.

There is just one pole: $z=a .$

The process generating system is asymptotically stable if $|a| <1$. 

Since $e(t)$ is a \gls{ssp} (by definition of \gls{wn}), when $|a| <1$ the steady-state output process $y(t)$ is a \gls{ssp}.

\textbf{Mean.}
Start from the time-domain representation and apply expectation to both sides:
\[
	\E[y(t)]=\E[a \cdot y(t-1)+e(t)] \implies \E[y(t)]=a \cdot \E[y(t-1)]+\E[e(t)]
\]
Thanks to stationarity $\E[y(t)]=\E[y(t-1)]=m_{y}$, so that $m_{y}=a \cdot m_{y}+m_{e}$.
Then:
$$
m_{e}=0 \implies  m_{y}=0
$$
\textbf{Variance.}
Let us compute
\[
	\gamma_{y}(0)=\E\left[\left(y(t)-m_{y}\right)^{2}\right]
\]
Remember that $m_{y}=0$. Start from $y(t)=a \cdot y(t-1)+e(t)$, take the square and apply operator $\E[\cdot]$ to both side:
\begin{align*}
	\gamma_{y}(0)&=\E\left[(y(t))^{2}\right]\\
	&=\E\left[(a \cdot y(t-1)+e(t))^{2}\right]\\
	&=a^{2} \E\left[y(t-1)^{2}\right]+\E\left[e(t)^{2}\right]+2 a \E[y(t-1) e(t)]\\
\end{align*}
Mid-terms evaluation:
\[
	2 a \E[y(t-1) e(t)]=0 \qquad \text{(we will show this later)}
\]
Indeed, by using the MA($\infty$) representation for $y(t-1)$ (AR process):
$$
y(t-1)=e(t-1)+a \cdot e(t-2)+a^{2} \cdot e(t-3)+a^{3} \cdot e(t-4)+\cdots
$$
we have that:
\[
	\E[e(t) y(t-1)]=\E\left[e(t) \cdot\left(e(t-1)+a \cdot e(t-2)+a^{2} \cdot e(t-3)+a^{3} \cdot e(t-4)+\cdots\right)\right]=0
\]
(all products give null contribution).


$\E\left[(y(t-1))^{2}\right]=\gamma_{y}(0)$ (thanks to stationarity)

$\E\left[(e(t))^{2}\right]=\lambda^{2}$

Hence,
\[
	\gamma_{y}(0)=a^{2} \gamma_{y}(0)+\lambda^{2} \implies \gamma_{y}(0)=\frac{\lambda^{2}}{1-a^{2}}
\]
\textbf{Covariance.}
Remember that $m_{y}=0$ and substitute $y(t)=a \cdot y(t-1)+e(t)$,
\begin{align*}
	\gamma_{y}(1)&=\E\left[\left(y(t)-m_{y}\right) \cdot\left(y(t-1)-m_{y}\right)\right]\\
	&=\E[y(t) y(t-1)]\\
	&=\E[(a \cdot y(t-1)+e(t))y(t-1)]\\
	&=a \cdot \E\left[(y(t-1))^{2}\right]+\E[e(t) y(t-1)]
\end{align*}

We already know that $\E\left[(y(t-1))^{2}\right]=\gamma_{y}(0)$, while as before $\E[e(t) y(t-1)]=0$.
$$
\gamma_{y}(1)=a \cdot \gamma_{y}(0)=a \cdot \frac{\lambda^{2}}{1-a^{2}}
$$
Arguing the same way,
\[
	\gamma_{y}(2)=a \cdot \gamma_{y}(1)=a^{2} \cdot \frac{\lambda^{2}}{1-a^{2}}
\]
Summary:
\begin{equation*}
	\begin{cases}
		\gamma_{y}(0)=\frac{\lambda^{2}}{1-a^{2}} \\
		\gamma_{y}(1)=\gamma_{y}(-1)=a \cdot \gamma_{y}(0) \\
		\gamma_{y}(2)=\gamma_{y}(-2)=a \cdot \gamma_{y}(1) \\
		\vdots
	\end{cases}
\end{equation*}

Recursive expression for $\gamma_{y}(\tau)$
$$
	\gamma_{y}(\tau)=a^{\tau} \cdot \frac{\lambda^{2}}{1-a^{2}}
$$
This result has been established for a generic AR($1$) process.
Those equations are called \emph{Yule-Walker equations}.

Grafical representation:

\fg{0.7}{Screenshot (14)}

\fg{0.7}{Screenshot (16)}

\subsection{ARMA processes}
$$
y(t)=a_{1} y(t-1)+\cdots+a_{m} y(t-m)+c_{0} e(t)+\cdots+c_{n} e(t-n)
$$
where $e(t) \sim \WN\left(0, \lambda^{2}\right)$.

\textbf{Mean.}
\begin{align*}
	\E[y(t)] &=\E\left[a_{1} y(t-1)+\cdots+a_{m} y(t-m)+c_{0} e(t)+\cdots+c_{n} e(t-n)\right] \\
	&=a_{1} \E[y(t-1)]+\cdots+a_{m} \E[y(t-m)]+c_{0} \E[e(t)]+\cdots+c_{n} \E[e(t-n)]
\end{align*}
$$
m_{y}=a_{1} m_{y}+\cdots+a_{m} m_{y}+c_{0} \cdot 0+\cdots+c_{n} \cdot 0
$$

By asintotical stability we can prove that $(1-a_1-\cdots-a_m)\neq0$, i.e. $m_{y}=0$.

\textbf{Covariance.}
\begin{align*}
	\gamma_{y}(0) = \E\left[y(t)^{2}\right]&=\E\left[\left(a_{1} y(t-1)+\cdots+a_{m} y(t-m)+c_{0} e(t)+\cdots+c_{n} e(t-n)\right)^{2}\right]\\
	&= a_{1}^{2} \E\left[y(t-1)^{2}\right]+a_{2}^{2} \E\left[y(t-2)^{2}\right]+2 a_{1} a_{2} \E[y(t-1) y(t-2)]+\cdots \\
	&\qquad+c_{0}^{2} \E\left[e(t)^{2}\right]+2 a_{1} c_{0} \E[y(t-1) e(t)]+\cdots
\end{align*}

Hence
$$
\gamma_{y}(0)=a_{1}^{2} \gamma_{y}(0)+a_{2}^{2} \gamma_{y}(0)+2 a_{1} a_{2} \gamma_{y}(1)+\cdots
$$

Then
\begin{align*}
	\E[y(t) y(t-1)]&=\E\left[\left(a_{1} y(t-1)+\cdots+a_{m} y(t-m)+c_{0} e(t)+\cdots+c_{n} e(t-n)\right) y(t-1)\right]= \\
	&=a_{1} \E\left[y(t-1)^{2}\right]+\cdots+c_{0} \E[e(t) y(t-1)]+\cdots
\end{align*}

Proceeding this way:
$$
\begin{cases}
	\gamma_{y}(0)=a_{1}^{2} \gamma_{y}(0)+a_{2}^{2} \gamma_{y}(0)+2 a_{1} a_{2} \gamma \\
	\gamma_{y}(1)=a_{1} \gamma_{y}(0)+\cdots+c_{0} \E[e(t) y(t-1)] \\
	\vdots \\
	\gamma_{y}(m-1)=a_{1} \gamma_{y}(m-2)+\cdots
\end{cases}
$$

$m$ variables and $m$ linear equations (\emph{Yule-walker equations} for an ARMA process).

Then, $\gamma_{y}(m), \gamma_{y}(m+1), \ldots$ can be recursevely computed from
$$
\gamma_{y}(0), \gamma_{y}(1), \ldots, \gamma_{y}(m-1)
$$
\begin{align*}
	\gamma_{y}(m)&=\E[y(t) y(t-m)] \\
	&=\E\left[\left(a_{1} y(t-1)+\ldots+a_{m} y(t-m)+c_{0} e(t)+\ldots+c_{n} e(t-n)\right) y(t-m)\right]=\ldots
\end{align*}
%!TEX root = ../main.tex

%!TEX root = ../main.tex

%!TEX root = ../main.tex
\section{Non-zero mean ARMA processes}\label{sec:non-zero-mean-arma}
Consider now $y(t)$ ARMA process generated as the steady-state output of a linear operator $W(z)$ that receive as input $e(t)$, where $e(t) \sim \WN\left(\mu, \lambda^{2}\right)$, with $\mu\neq0$.

\begin{theorem}[Gain theorem]\label{thm:gain-theorem}
	The steady-state output is constant and it holds that:
	\[
		\E[y(t)]=m_{y}=W(z)|_{z=1} \cdot \mu
	\]
\end{theorem}

Indeed:
\begin{align*}
		\E[y(t)] &=\E\left[a_{1} y(t-1)+\cdots+a_{m} y(t-m)+c_{0} e(t)+\cdots+c_{n} e(t-n)\right] \\
		&=a_{1} \E[y(t-1)]+\cdots+a_{m} \E[y(t-m)]+c_{0} \E[e(t)]+\cdots+c_{n} \E[e(t-n)]
\end{align*}
$$
m_{y}=a_{1} m_{y}+\cdots+a_{m} m_{y}+c_{0} \cdot \mu+\cdots+c_{n} \cdot \mu
$$
i.e. $m_{y}=\frac{c_{0}+c_{1}+\cdots+c_{n}}{1-a_{1}-\cdots-a_{m}} \cdot \mu=W(1) \cdot \mu$.
	
Define two new processes (\emph{unbiased} processes)
$$
\begin{cases}
	\tilde{y}(t)=y(t)-m_{y} \quad\forall t\\
	\tilde{e}(t)=e(t)-m_{e} \quad\forall t
\end{cases}
\rightarrow \begin{array}{l}
	\E[\tilde{y}(t)]=\E[y(t)]-m_{y}=0 \\
	\E[\tilde{e}(t)]=\E[e(t)]-m_{e}=0
\end{array}
$$
\begin{align*}
	\tilde{y}(t)&= y(t)-m_{y}\\
	&= a_{1} y(t-1)+\cdots+a_{m} y(t-m)+c_{0} e(t)+\cdots+c_{n} e(t-n)-m_{y} \\
	&= a_{1}\left(\bar{y}(t-1)+m_{y}\right)+\cdots+a_{m e}\left(\bar{y}(t-m)+m_{y}\right)+\\
	&\qquad +c_{0}\left(\tilde{e}(t)+m_{e}\right)+\cdots+c_{n}\left(\bar{e}(t-n)+m_{e}\right)-m_{y} \\
	&= a_{1} \tilde{y}(t-1)+\cdots+a_{m} \tilde{y}(t-m)+c_{0} \tilde{e}(t)+\cdots+c_{n} \tilde{e}(t-n) \\
	&\qquad \underbrace{-\left(1-a_{1}-\cdots-a_{m}\right) m_{y}+\left(c_{0}+\cdots c_{n}\right) m_{e}}_{=0\text { since } m_{y}=\frac{c_{0}+c_{1}+\cdots+c_{n}}{1-a_{1}-\cdots-a_{m e}} m_{e}=W(1) \cdot \mu} \\
\end{align*}
Hence,
$$
\tilde{y}(t)=a_{1} \tilde{y}(t-1)+\cdots+a_{m} \tilde{y}(t-m)+c_{0} \tilde{e}(t)+\cdots+c_{n} \tilde{e}(t-n)
$$
where $\tilde{e}(t)\sim \WN\left(0, \lambda^{2}\right)$, is Standard zero mean ARMA process.

$\tilde{y}(t)$ is the steady-state solution to $W(z)=\frac{A(z)}{C(z)}$ (same transfer function as before) fed by $\tilde{e}(t)$.

Moreover,
$$
\gamma_{y}(\tau)=\E\left[\left(y(t)-m_{y}\right) \cdot\left(y(t-\tau)-m_{y}\right)\right]=\E[(\tilde{y}(t)) \cdot(\tilde{y}(t-\tau))]=\gamma_{\tilde{y}}(\tau)
$$

\textbf{Remark.} We cannot drop $m_{y} \neq 0$.

$$\quad \gamma_{y}(\tau) \neq \E[y(t) \cdot y(t-\tau)]$$

We can see this method graphically:

\fg{0.6}{Screenshot (19)}

\section{ARMAX(ARX) processes}
ARMA processes with eXogenous input.

%% I/O systems
A process $y(t)$, generated by a remote \gls{wn} input $e(t)$ and by an exogenous (measurable) input $u(t)$, is an ARMAX process if:
\begin{align*}
	y(t)&=a_{1} y(t-1)+a_{2} y(t-2)+\cdots+a_{m} y(t-m)+ &\text{AR($m$) part}\\
	&\qquad+c_{0} e(t)+c_{1} e(t-1)+\cdots+c_{n} e(t-n)+ &\text{MA($n$) part} \\
	&\qquad+b_{0} u(t-k)+b_{1} u(t-k-1)+\cdots+b_{p} u(t-k-p)  &\text{X($k,p$) part}
\end{align*}

ARMAX($m,p,n,k$) denotes an ARMAX process of orders $m,p,n$ with input delay between input $u(t)$ and ouput $y(t)$ equal to $k$.

ARX($m,p$) equals to ARMAX($m,p,0$).

Then:
\begin{align*}
	y(t) &= \frac{\left(b_{0}+b_{1} z^{-1}+\cdots+b_{p} z^{-p}\right) z^{-k}}{\left(1-a_{1} z^{-1}-a_{2} z^{-2}-\cdots-a_{m} z^{-m}\right)} u(t)+\frac{\left(c_{0}+c_{1} z^{-1}+\cdots+c_{n} z^{-n}\right)}{\left(1-a_{1} z^{-1}-a_{2} z^{-2}-\cdots-a_{m} z^{-m}\right)} e(t)\\
	&=\frac{B(z) z^{-k}}{A(z)} u(t)+\frac{C(z)}{A(z)} e(t)
\end{align*}
where:
\begin{align*}
	B(z) &= \left(b_{0}+b_{1} z^{-1}+\cdots+b_{p} z^{-p}\right)\\
	C(z) &= \left(c_{0}+c_{1} z^{-1}+\cdots+c_{n} z^{-n}\right)\\
	A(z) &= \left(1-a_{1} z^{-1}-a_{2} z^{-2}-\cdots-a_{m} z^{-m}\right)
\end{align*}

Both $\frac{B(z) z^{-k}}{A(z)}$ and $\frac{C(z)}{A(z)}$ are transfer functions.

An ARMAX processes can be seen as the sum of a deterministic part (output of $\frac{B(z) z^{-k}}{A(z)}$ fed by $u(t)$) and a stochastic part (output of $\frac{C(z)}{A(z)}$ fed by $e(t)$, ARMA).

In the ARX process $C(z)=1$.

\section{Analysis in the Frequency Domain}
\begin{definition}
	The spectral density of a \gls{ssp} $y(t)$ (also called the \textbf{spectrum} of $y(t)$ ) is defined as:
	\[
		\Gamma_{y}(\omega)=\sum_{t=-\infty}^{+\infty} \gamma_{y}(\tau) \cdot e^{-j \omega \tau}.
	\]
\end{definition}

In other words, $\Gamma_y(\omega)$ is defined as the Fourier transform of the covariance function.

Properties of $\Gamma_{y}(\omega)$:
\begin{enumerate}
	\item is a real function of the real variable $\omega$, $\Im\left(\Gamma_{y}(\omega)\right)=0 \quad \forall \omega \in \Re$
	\item is a positive function,
	$$
		\Gamma_{y}(\omega) \geq 0 \quad \forall \omega \in \mathbb{Y}^{\mathrm{r}}
	$$
	\item is an even function,
	$$
		\Gamma_{y}(\omega)=\Gamma_{y}(-\omega) \quad \forall \omega \in \Re
	$$
	\item is $2\pi$-periodic: $\Gamma_{y}(\omega)=\Gamma_{y}(\omega+k \cdot 2 \pi) \quad \forall \omega \in \Re, \forall k \in \mathbb{Z}$.
\end{enumerate}

\textbf{Observation.}
As a consequence of 3) and 4), we will plot the spectral density in the interval $[0, \pi]$.

\textbf{Example.}
Let us consider $e(t) \sim \WN\left(\mu, \lambda^{2}\right)$. 

Covariance function:
\[
	\gamma_{e}(\tau)= \begin{cases}\lambda^{2} & \text { if } \tau=0 \\ 0 & \text { if } \tau \neq 0\end{cases}
\]


Spectral density:
\[
	\Gamma_{e}(\omega) =\sum_{\tau=-\infty}^{+\infty} \gamma_{e}(\tau) \cdot e^{-j \omega \tau}=\gamma_{e}(0) e^{-j \omega 0}+\gamma_{e}(1) e^{-j \omega}+\gamma_{e}(-1) e^{j \omega}+\cdots=\gamma_{e}(0)=\lambda^{2}
\]
\gls{wn} have constant and equal to $\lambda^{2}$ spectral density.

\textbf{Example} (MA(1) process)

$y(t)=e(t)+c \cdot e(t-1), c \in \mathcal{R}$ (real coefficient)
$e(t)\sim \WN(0,\lambda^{2} )$

\begin{align*}
	&\gamma_{y}(0)=\left(1^{2}+c^{2}\right) \cdot \lambda^{2}=(1+c^{2})\lambda^{2}\\
	&\gamma_{y}(1)=(1 \cdot c) \cdot \lambda^{2}=c\lambda^{2} \\
	&\gamma_{y}(\tau)=0 \text { when } \tau=\ldots, \pm 3, \pm 4, \ldots .
\end{align*}
Spectral density (via the definition):
\begin{align*}
	\Gamma_{y}(\omega)&=\sum_{\tau=-\infty}^{+\infty} \gamma_{y}(\tau) \cdot e^{-j \omega \tau}= \\
	&\text{(only $=\gamma_{y}(0), \gamma_{y}(\pm 1)\left(e^{-j \omega}\right)$ are not null)} \\
	&=\gamma_{y}(0)+\gamma_{y}(1)\left(e^{-j \omega}\right)+\gamma_{y}(-1)\left(e^{+j \omega}\right)+0= \\
	&=1+c^{2}+c\left(e^{-j \omega}+e^{j \omega}\right)
\end{align*}

Euler representation of the exponential
$$
e^{-j \omega}+e^{+j \omega}=\cos (\omega)-j \sin (\omega)+\cos (\omega)+j \sin (\omega)=2 \cos (\omega)
$$

We have:
$$\Gamma_y(\omega)=1+c^{2}+2 c \cos (\omega).$$
which is real, even and periodic with period $2 \pi$.

Let the process $y(t)$ be the steady-state output of an asymptotically stable digital filter fed by an \gls{ssp}, i.e. $y(t)=F(z) v(t)$.

Then, the following formula for the spectral density of $y(t)$ holds:
$$\Gamma_{y}(\omega)=|F\left(e^{j \omega}\right)|^{2} \cdot \Gamma_{v}(\omega)$$

\begin{itemize}
	\item $\Gamma_{y}(\omega)$ is the output spectral density;
	\item $\left|F\left(e^{j \omega}\right)\right|^{2}$ is the square absolute value of the filter transfer function evaluate for $z=e^{j \omega}$ (filter frequency response);
	\item $\Gamma_{v}(\omega)$ is the input spectral density.
\end{itemize}

If the input $v(t)$ is a \gls{wn} with variance $\lambda^{2}$, then:
$$
\Gamma_{y}(\omega)=\left|F\left(e^{j \omega}\right)\right|^{2} \lambda^{2}
$$

\textbf{Example.} (MA(1) process)

$y(t)=e(t)+c \cdot e(t-1), c \in \mathcal{R}$ (real coefficient)
$e(t)\sim \WN(0,\lambda^{2} )$

$$
y(t)=\left(1+c z^{-1}\right) e(t)
$$
$$
\Gamma_{y}(\omega)=\left|1+c e^{-j \omega}\right|^{2} \cdot \lambda^{2}
$$
Recalling that:
\begin{itemize}
	\item $|a+j b|^{2}=(a+j b)(a-j b)=a^{2}+b^{2}$
	\item $1+c e^{-j \omega}=1+c \cos (-\omega)+j \sin (-\omega)=1+c \cos (\omega)-j \sin (\omega)$ $1+c e^{+j \omega}=1+c \cos (\omega)+j \sin (\omega)$
\end{itemize}

$$
\Gamma_{y}(\omega)=\left(1+c e^{-j \omega}\right)\left(1+c e^{+j \omega}\right)\cdot \lambda^{2}=(1+c^{2}+c\left(e^{j \omega}+e^{-j \omega}\right))\cdot \lambda^{2}=(1+c^{2}+2 c \cos (\omega))\cdot \lambda^{2}
$$

\fg{0.7}{Screenshot (20)}

Moreover:
\begin{align*}
	|1+c e^{-j \omega}|^2&=\left(1+c e^{-j \omega}\right)\left(1+c e^{+j \omega}\right)\\
	&=\underbrace{\left(1+c z^{-1}\right)}_\text{where $z=e^{j \omega}$}\underbrace{\left(1+c z^{-1}\right)}_\text{where $z=e^{-j \omega}$}
\end{align*}
\begin{align*}
	W(z)=\frac{C(z)}{A(z)}=\frac{(1+q_1z^{-1})\cdots(1+q_nz^{-1})}{(1+p_1z^{-1})\cdots(1+p_mz^{-1})}
\end{align*}
where $q_i$ are the zeros and $p_i$ the poles.

\begin{align*}
	\left|W\left(e^{j \omega}\right)\right|^{2}&=\frac{|1+q_1e^{-j \omega}|^2\cdots|1+q_ne^{-j \omega}|^2}{|1+p_1e^{-j \omega}|^2\cdots|1+p_me^{-j \omega}|^2}\\
	&=\frac{(1+q_1e^{-j \omega})\cdots(1+q_ne^{-j \omega})}{(1+p_1e^{-j \omega})\cdots(1+p_me^{-j \omega})}\cdot\frac{(1+q_1e^{j \omega})\cdots(1+q_ne^{j \omega})}{(1+p_1e^{j \omega})\cdots(1+p_me^{j \omega})}\\
			&=W\left(e^{j \omega}\right)\cdot W\left(e^{-j \omega}\right)
\end{align*}
%!TEX root = ../main.tex

%!TEX root = ../main.tex

%!TEX root = ../main.tex

%!TEX root = ../main.tex

%!TEX root = ../main.tex

%!TEX root = ../main.tex

%!TEX root = ../main.tex

%!TEX root = ../main.tex
\section{Optimal prediction for ARMAX processes}

\[
	y(t)=\underbrace{\frac{B(z)}{A(z)} u(t-d)}_{\text{deterministic}}+
	\underbrace{\frac{C(z)}{A(z)} e(t)}_{\text{stochastic}}\qquad e(t)\sim \WN(0,\lambda^2 )
\]
Without loss of generality, non-zero mean can be always incorporated in $u$. Available information:
\begin{gather*}
	y(t),y(t-1),\ldots \\
	u(t),u(t-1),\ldots
\end{gather*}
\textbf{Hypothesis:}
\begin{itemize}
	\item $\frac{C(z)}{A(z)}e(t)$ is a canonical representation (otherwise compute it);
	\item either $u$ is completely known (pre-deterministic signal) from $t=-\infty$ up to $t=+\infty$ or $d\geq k$ (delay bigger than prediction error).
\end{itemize}
Let 
$$
	z(t)=y(t)-\frac{B(z)}{A(z)} u(t-d)
$$
Then, $z(t)=\frac{C(z)}{A(z)} e(t)$ i.e. it is an ARMA process such that:
$$
	\frac{C(z)}{A(z)}=E(z)+z^{-k} \frac{F(z)}{A(z)} \quad\text{($k$-steps division between $C(z)$ and $A(z)$)}
$$
Then:
$$
	\hat{z}(t+k \mid t)=\frac{F(z)}{C(z)} z(t)
$$
$$
	y(t+k)=\frac{B(z)}{A(z)} u(t+k-d)+z(t+k)
$$
The first part is deterministically known, and hence can be trivially predicted.
\begin{align*}
	\hat{y}(t+k \mid t)&=\frac{B(z)}{A(z)} u(t+k-d)+\hat{z}(t+k \mid t) \\
	&=\frac{B(z)}{A(z)} u(t+k-d)+\frac{F(z)}{C(z)} z(t) \\
	& =\frac{B(z)}{A(z)} u(t+k-d)+\frac{F(z)}{C(z)}\Bigg(y(t)-\frac{B(z)}{A(z)} \underbrace{u(t-d)}_{z^{-k}u(t+k-d)}\Bigg) \\
	&=\frac{B(z)}{C(z)} \cdot\Bigg(\underbrace{\frac{C(z)}{A(z)}-\frac{z^{-k} F(z)}{A(z)}}_{E(z)}\Bigg) u(t+k-d)+\frac{F(z)}{C(z)} y(t)
\end{align*}
$$
	\boxed{\hat{y}(t+k \mid t) =\frac{B(z) E(z)}{C(z)} \underbrace{u(t+k-d)}_{\text{known}}+\frac{F(z)}{C(z)} y(t)}
$$

\chapter{Model Identification}

Up to now, we considered models and studied their properties: covariance and spectrum computation, prediction.

But where does the model come from?

Model identification: retrieve a suitable model from experiments on the real system.

\fg{0.4}{Screenshot (25)}

\textbf{Identification problem:} define an automatic procedure to find a model for $S$ based on available (input/output or time series) data.

%\fg{0.7}{Screenshot (26)}
% \begin{figure}[htpb]
% 	\centering
% 	\begin{tikzpicture}

% 	% place nodes
% 		\node [sum] (sum) at (0,0) {};
% 		\node [block,above=1cm of sum]  (ca) {$\frac{C(z)}{A(z)}$};
% 		\node [block,left =1cm of sum]  (ba) {$\frac{B(z)}{A(z)}$};
% 		%\node [above left = 0cm and 2.5cm of ba] {$\begin{array}{c} y(1),y(2),\ldots,y(N) \\ u(1),u(2),\ldots,u(N) \end{array}\implies$};

% 		% connect nodes
% 		\draw[-stealth] (ba.east) -- (sum.west) node[near end,above]{$+$};
% 		\draw[-stealth] (ca.south) -- (sum.north) node[near end,left]{$+$};
% 		\draw[-stealth] (sum.east) -- ++(2,0) node[midway,above]{$y(t)$};
% 		\draw[stealth-] (ca.west) -- ++(-2,0) node[midway, above]{$e(t)$};
% 		\draw[stealth-] (ba.west) -- ++(-2,0) node[midway, above]{$u(t-d)$};

% 	\end{tikzpicture}
% \end{figure}
% \FloatBarrier

\section{Parametric model identification}

First we select a \textbf{parametric model class} $\Mc(\theta)$, where $\theta$ is the \textbf{vector of parameters} (each different $\theta$ corresponds to a different model in that class).

For example the family of ARMAX models whose coefficients are polynomials is:
$$
	\Mc(\theta)=\left\{y(t)=\frac{B(z, \theta)}{A(z, \theta)} u(t-d)+\frac{C(z, \theta)}{A(z, \theta)} e(t), \quad e(t) \sim \WN(0, \lambda^{2}), \theta\in\Theta\right\}
$$
where:
\begin{align*}
	A(z, \theta)&=1-a_{1}(\theta) z^{-1}-\cdots-a_{m}(\theta) z^{-m} \\
	B(z, \theta)&=b_{1}(\theta)+b_{2}(\theta) z^{-1}+\cdots+b_{p}(\theta) z^{-p}\\
	C(z, \theta)&=c_0(\theta)+c_{1}(\theta) z^{-1}+\cdots+c_{n}(\theta) z^{-n}
\end{align*}

We will talk of \textbf{black-box identification} when no knowledge on the system is available and the model structure must be found from data only.\\
$\theta$ is directly the vector of coefficients of $A(z, \theta),B(z, \theta),C(z, \theta)$.
$$
	\theta=[a_{1},\ldots,a_{m},b_{1},\ldots,b_{p},c_{1},\ldots,c_{n}]\transpose
$$
In all other cases: \textbf{grey-box identification}.

We try to incorporate some \emph{a priori} information about the real $S$ in the model class. $\theta$ may have some physical interpretation.

\begin{example}
$$
	\Mc(\theta) = \left\{ y(t)=\frac{b+b^{2} z^{-1}}{1-a z^{-1}} u(t-d)+\frac{1+a z^{-1}}{1-a z^{-1}} e(t) \right\}  
$$
Here the parameter vector is given by $\theta=\begin{bmatrix}a,b\end{bmatrix}\transpose$ only.
\end{example}

\begin{obs}
$\lambda^2$ is a parameter which needs to be identified too, however, $\lambda^2$ is much less important than other parameters. So, we will indicate by $\theta$ the vector of \emph{important} parameters and keep $\lambda^2$ aside.
\end{obs}

$\Theta$ is the set of admissible values for the parameter vector $\theta$.

It incorporates a-priori information on the possible values for the parameters. In the black-box case $\Theta$ is as free as possible.

As we will see, to perform identification we will rely on the theory of prediction. Hence, we will assume the following:

\boxedText{For every $\theta \in \Theta$, the stochastic part of $\Mc(\theta)$ (i.e. the part depending on the white noise $e(t) \sim \WN(0, \lambda^{2})$) is \textbf{canonical} and has \textbf{no zeros on the unit circle}.}

The requirement that there are no zeros on the unit circle instead poses some limitations on the systems we can identify. However:
\begin{itemize}
	\item zeros on the unit circle are not usually required to model the behavior of a given system;
	\item the behavior of models with zeros on the unit circle can be approximated by models with zeros \emph{close} to the unit circle.
\end{itemize} 

\section{Prediction Error Minimization (PEM) Identification}

Paradigm to map data into a value of $\overline{\theta}$.
$$
	D^N=\left\{\overline{y(1)},\ldots,\overline{y(N)},\overline{u(1)},\ldots,\overline{u(N)}\right\}
$$
$D^N$ is a finite sequence of real numbers, observations of $y$ and $u$ over some horizon. $N$ is the length of the dataset.
% \fg{0.2}{Screenshot (27)}
How to compare $\overline{y(1)},\ldots,\overline{y(N)}$ with $y(1,s),\ldots,y(N,s)$?
\fg{0.6}{Screenshot (28)}
The idea of the \gls{pem} is that we move \textbf{from stochastic models to predictive models.}
\[
	\Mc(\theta) \to \hat{\Mc}(\theta)
\]
For example this is an ARMAX in $\Mc(\theta)$ for a given value of $\theta$:
\begin{align*}
	\Mc(\theta)&:\quad y(t)=\frac{B(z, \theta)}{A(z, \theta)} u(t-d)+\frac{C(z, \theta)}{A(z, \theta)} e(t)\\
	&\qquad\downarrow\\
	\hat{\Mc}(\theta)&: \quad \hat{y}(t \mid t-1) =\frac{B(z,\theta) E(z,\theta)}{C(z,\theta)} u(t-d)+\frac{F(z,\theta)}{C(z,\theta)} y(t-1) \qquad \text{one step predictor}
\end{align*}
% \fg[The \gls{pem} identification scheme.]{0.7}{Screenshot (29)}
% $$y(t)=\underbrace{\frac{B(z)}{A(z)} u(t+k-d)}_{\substack{\text{This part of the process}\\ 
% 		\text{is deterministically}\\
% 		\text{known, and hence can}\\
% 		\text{be trivially predicted}}} +z(t+k)$$
%!TEX root = ../main.tex

%!TEX root = ../main.tex

%!TEX root = ../main.tex

%!TEX root = ../main.tex

%!TEX root = ../main.tex

%!TEX root = ../main.tex

%!TEX root = ../main.tex

%!TEX root = ../main.tex

%!TEX root = ../main.tex

%!TEX root = ../main.tex

%!TEX root = ../main.tex

%!TEX root = ../main.tex

%!TEX root = ../main.tex

%!TEX root = ../main.tex

%!TEX root = ../main.tex

%!TEX root = ../main.tex

%!TEX root = ../main.tex

%!TEX root = ../main.tex

%!TEX root = ../main.tex

%!TEX root = ../main.tex

%!TEX root = ../main.tex

%!TEX root = ../main.tex

%!TEX root = ../main.tex

%!TEX root = ../main.tex

%!TEX root = ../main.tex

%!TEX root = ../main.tex

%!TEX root = ../main.tex

%!TEX root = ../main.tex

%!TEX root = ../main.tex

%!TEX root = ../main.tex

%!TEX root = ../main.tex

%!TEX root = ../main.tex

%!TEX root = ../main.tex

%!TEX root = ../main.tex

%!TEX root = ../main.tex

%!TEX root = ../main.tex

%!TEX root = ../main.tex

%!TEX root = ../main.tex

%!TEX root = ../main.tex

%!TEX root = ../main.tex

%!TEX root = ../main.tex

%!TEX root = ../main.tex

%!TEX root = ../main.tex


\end{document}