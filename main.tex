\documentclass[10pt,a4paper,twoside,openright]{book}

%%%%%%%%%%%%%%%%%%%%%%%%%%%%%%%%%%%%%%%%%
% Template Dispense
% Autore: Teo Bucci
%%%%%%%%%%%%%%%%%%%%%%%%%%%%%%%%%%%%%%%%%

%----------------------------------------------------------------------------------------
%	FONTS AND LANGUAGE
%----------------------------------------------------------------------------------------

\usepackage[T1]{fontenc} % Use 8-bit encoding that has 256 glyphs
\usepackage[utf8]{inputenc} % Required for including letters with accents
\usepackage[english]{babel} % Italian language/hyphenation

%----------------------------------------------------------------------------------------
%	PACKAGES
%----------------------------------------------------------------------------------------

\usepackage{amsmath,amssymb,amsthm} % amssymb carica anche amsfonts

\usepackage{dsfont} % per funzione indicatrice
\newcommand{\Ind}{\mathds{1}}
%\usepackage{tikz-among-us}

\usepackage{booktabs}
\usepackage{pgfplots}
\usepackage{tikz}
\usetikzlibrary{positioning,shapes.misc,intersections,shapes.symbols}
\usepackage{mathdots}
\usepackage{cancel}
\usepackage{color}
\usepackage{siunitx}
\usepackage{array}
\usepackage{multirow}
\usepackage{makecell}
\usepackage{tabularx}
%\usepackage{booktabs}
\usepackage{caption}\captionsetup{belowskip=12pt,aboveskip=4pt}
\usepackage{subcaption}
\usepackage{placeins} % The placeins package gives the command \FloatBarrier, which will make sure any floats will be put in before this point
\usepackage{flafter}  % The flafter package ensures that floats don't appear until after they appear in the code.

\usepackage{import}
\usepackage{pdfpages}
\usepackage{transparent}
\usepackage{xcolor}
\usepackage{graphicx}
\graphicspath{ {./images/} }
\usepackage{float}
\usepackage[italian]{varioref}

\newcommand{\fg}[3][\relax]{%
  \begin{figure}[H]%[htp]%
    \centering
    \captionsetup{width=0.7\textwidth}
      \includegraphics[width = #2\textwidth]{#3}%
      \ifx\relax#1\else\caption{#1}\fi
      \label{#3}
  \end{figure}%
  \FloatBarrier%
}

%----------------------------------------------------------------------------------------
%	PARAGRAFI, INTERLINEA E MARGINE
%----------------------------------------------------------------------------------------

\usepackage[none]{hyphenat} % Per non far andare a capo le parole con il trattino

\emergencystretch 3em % Per evitare che il testo vada oltre i margini

% \parindent 0ex % Toglie intendamento paragrafi, e' incluso nel pacchetto \parskip
% \setlength{\parindent}{4em} % Variante di quello sopra

% \setlength{\parskip}{\baselineskip} % Cambia spazio tra paragrafi (posso mettere anche 1em) inclusa la table of contents, pertanto uso il comando che segue

\usepackage[skip=0.2\baselineskip+2pt]{parskip}

% \renewcommand{\baselinestretch}{1.5} % Cambia interlinea

%----------------------------------------------------------------------------------------
%	HEADERS AND FOOTERS
%----------------------------------------------------------------------------------------

\usepackage{fancyhdr}


\fancypagestyle{toc}{%
\fancyhf{}%
\fancyfoot[C]{\thepage}%
\renewcommand{\headrulewidth}{0pt}%
\renewcommand{\footrulewidth}{0pt}
}

\fancypagestyle{fancy}{%
\fancyhf{}%
\fancyhead[RE]{\nouppercase{\leftmark}}%
\fancyhead[LO]{\nouppercase{\rightmark}}%
\fancyhead[LE,RO]{\thepage}%
\renewcommand{\footrulewidth}{0pt}%
\renewcommand{\headrulewidth}{0.4pt}
}

% Removes the header from odd empty pages at the end of chapters
\makeatletter
\renewcommand{\cleardoublepage}{
\clearpage\ifodd\c@page\else
\hbox{}
\vspace*{\fill}
\thispagestyle{empty}
\newpage
\fi}

%----------------------------------------------------------------------------------------
%	COMANDI PERSONALIZZATI
%----------------------------------------------------------------------------------------

\newcommand{\Tau}{\mathcal{T}}
\DeclareMathOperator{\sgn}{sgn}
\newcommand{\degree}{^\circ\text{C}} % SIMBOLO GRADI
\newcommand{\notimplies}{\mathrel{{\ooalign{\hidewidth$\not\phantom{=}$\hidewidth\cr$\implies$}}}}
\renewcommand{\qed}{\tag*{$\blacksquare$}}
\usepackage{xspace}
\newcommand{\latex}{\LaTeX\xspace}
\newcommand{\tex}{\TeX\xspace}
\newcommand{\questeq}{\overset{?}{=}} % è vero che?
\newcommand{\complementary}{^{\mathrm{C}}}
\newcommand{\transpose}{^{\mathrm{T}}}
\renewcommand{\emptyset}{\varnothing} % simbolo insieme vuoto
\newcommand{\Bot}{\perp \!\!\! \perp} % indipendenza
\renewcommand{\tilde}{\widetilde}
\renewcommand{\hat}{\widehat}
\renewcommand{\theta}{\vartheta}
\newcommand{\boxedText}[1]{\noindent\fbox{\parbox{\textwidth-0.5cm}{#1}}}

\usepackage{mathtools} % Serve per i comandi dopo
\DeclarePairedDelimiter{\abs}{\left\lvert}{\right\rvert}
\DeclarePairedDelimiter{\norm}{\left\lVert}{\right\rVert}
\DeclarePairedDelimiter{\sca}{\left\langle}{\right\rangle}

\newcommand{\fou}[1]{\mathcal{F}\left\{#1\right\}}
\newcommand{\ifou}[1]{\mathcal{F}^{-1}\left\{#1\right\}}
\newcommand{\Mc}{\mathcal{M}}
\newcommand{\E}{\mathbb{E}}
\newcommand{\R}{\mathbb{R}}
\newcommand{\N}{\mathbb{N}}
\newcommand{\C}{\mathbb{C}}
\newcommand{\Z}{\mathbb{Z}}
%\DeclarePairedDelimiter{\fou}{\mathcal{F}}{}
%\DeclarePairedDelimiter{\ifou}{\mathcal{F}^{-1}\left\{}{\right\}}

%----------------------------------------------------------------------------------------
%	APPENDICE
%----------------------------------------------------------------------------------------

\usepackage[toc,page]{appendix}

%----------------------------------------------------------------------------------------
%	SIMBOLI CARTE DA GIOCO
%
% Sono i seguenti:
%   \varheartsuit
%   \vardiamondsuit
%   \clubsuit
%   \spadesuit
%----------------------------------------------------------------------------------------

\DeclareSymbolFont{extraup}{U}{zavm}{m}{n}
\DeclareMathSymbol{\varheartsuit}{\mathalpha}{extraup}{86}
\DeclareMathSymbol{\vardiamondsuit}{\mathalpha}{extraup}{87}

%----------------------------------------------------------------------------------------

\definecolor{grey245}{RGB}{245,245,245}

\newtheoremstyle{blacknumbox} % Theorem style name
{0pt}% Space above
{0pt}% Space below
{\normalfont}% Body font
{}% Indent amount
{\bf\scshape}% Theorem head font --- {\small\bf}
{.\;}% Punctuation after theorem head
{0.25em}% Space after theorem head
{\small\thmname{#1}\nobreakspace\thmnumber{\@ifnotempty{#1}{}\@upn{#2}}% Theorem text (e.g. Theorem 2.1)
%{\small\thmname{#1}% Theorem text (e.g. Theorem)
\thmnote{\nobreakspace\the\thm@notefont\normalfont\bfseries---\nobreakspace#3}}% Optional theorem note

% Per gli unnumbered tolgo il \nobreakspace subito dopo {\small\thmname{#1} perché altrimenti c'è uno spazio tra Teorema e il ".", lo spazio lo voglio solo se sono numerati per distanziare Teorema e "(2.1)"
\newtheoremstyle{unnumbered} % Theorem style name
{0pt}% Space above
{0pt}% Space below
{\normalfont}% Body font
{}% Indent amount
{\bf\scshape}% Theorem head font --- {\small\bf}
{.\;}% Punctuation after theorem head
{0.25em}% Space after theorem head
{\small\thmname{#1}\thmnumber{\@ifnotempty{#1}{}\@upn{#2}}% Theorem text (e.g. Theorem 2.1)
%{\small\thmname{#1}% Theorem text (e.g. Theorem)
\thmnote{\nobreakspace\the\thm@notefont\normalfont\bfseries---\nobreakspace#3}}% Optional theorem note

\newcounter{dummy} 
\numberwithin{dummy}{chapter}

\theoremstyle{blacknumbox}
\newtheorem{definitionT}[dummy]{Definition}
\newtheorem{theoremT}[dummy]{Theorem}

\theoremstyle{unnumbered}
\newtheorem*{ossT}{Observation}
\newtheorem*{exT}{Example}
\newtheorem*{dimT}{Proof}

\RequirePackage[framemethod=default]{mdframed} % Required for creating the theorem, definition, exercise and corollary boxes

% orange box
\newmdenv[skipabove=7pt,
skipbelow=7pt,
rightline=false,
leftline=true,
topline=false,
bottomline=false,
linecolor=orange,
backgroundcolor=orange!5,
innerleftmargin=5pt,
innerrightmargin=5pt,
innertopmargin=5pt,
leftmargin=0cm,
rightmargin=0cm,
linewidth=2pt,
innerbottommargin=5pt]{oBox}

% green box
\newmdenv[skipabove=7pt,
skipbelow=7pt,
rightline=false,
leftline=true,
topline=false,
bottomline=false,
linecolor=green,
backgroundcolor=green!0,
innerleftmargin=5pt,
innerrightmargin=5pt,
innertopmargin=5pt,
leftmargin=0cm,
rightmargin=0cm,
linewidth=2pt,
innerbottommargin=5pt]{gBox}

% blue box
\newmdenv[skipabove=7pt,
skipbelow=7pt,
rightline=false,
leftline=true,
topline=false,
bottomline=false,
linecolor=blue,
backgroundcolor=blue!0,
innerleftmargin=5pt,
innerrightmargin=5pt,
innertopmargin=5pt,
leftmargin=0cm,
rightmargin=0cm,
linewidth=2pt,
innerbottommargin=5pt]{bBox}

% dim box
\newmdenv[skipabove=7pt,
skipbelow=7pt,
rightline=false,
leftline=true,
topline=false,
bottomline=false,
linecolor=black,
backgroundcolor=grey245!0,
innerleftmargin=5pt,
innerrightmargin=5pt,
innertopmargin=5pt,
leftmargin=0cm,
rightmargin=0cm,
linewidth=2pt,
innerbottommargin=5pt]{dimBox}

\newenvironment{theorem}{\begin{gBox}\begin{theoremT}}{\end{theoremT}\end{gBox}}	
\newenvironment{definition}{\begin{bBox}\begin{definitionT}}{\end{definitionT}\end{bBox}}	
\newenvironment{obs}{\begin{dimBox}\begin{ossT}}{\end{ossT}\end{dimBox}}
\newenvironment{example}{\begin{dimBox}\begin{exT}}{\end{exT}\end{dimBox}}
\newenvironment{dimostrazione}{\begin{dimBox}\begin{dimT}}{\[\qed\]\end{dimT}\end{dimBox}}

%----------------------------------------------------------------------------------------
%	INDICE INIZIALE
%----------------------------------------------------------------------------------------

\setcounter{secnumdepth}{3} % DI DEFAULT LE SUBSUBSECTION NON SONO NUMERATE, COSÌ SÌ
\setcounter{tocdepth}{2} % FISSA LA PROFONDITÀ DELLE COSE MOSTRATE NELL'INDICE

%\usepackage[hidelinks,bookmarksnumbered]{hyperref} % Rende l'indice interattivo e hidelinks nasconde il bordo rosso dai riferimenti

\usepackage{bookmark}% loads hyperref too
    \hypersetup{
        %pdftitle={Fundamentos de C\'alculo},
        %pdfsubject={C\'alculo diferencial},
        bookmarksnumbered=true,
        bookmarksopen=true,
        bookmarksopenlevel=1,
        hidelinks,% remove border and color
        pdfstartview=Fit, % Fits the page to the window.
        pdfpagemode=UseOutlines, %Determines how the file is opening in Acrobat; the possibilities are UseNone, UseThumbs (show thumbnails), UseOutlines (show bookmarks), FullScreen, UseOC (PDF 1.5), and UseAttachments (PDF 1.6). If no mode if explicitly chosen, but the bookmarks option is set, UseOutlines is used.
    }

\usepackage{glossaries} % certain packages that must be loaded before glossaries, if they are required: hyperref, babel, polyglossia, inputenc and fontenc
\setacronymstyle{long-short}

% Questo comando e quello dopo servono per avere il comando \tocless da mettere prima di una sezione che non voglio far apparire nell'indice
\newcommand{\nocontentsline}[3]{}
\newcommand{\tocless}[2]{\bgroup\let\addcontentsline=\nocontentsline#1{#2}\egroup}

%\usepackage{showframe}
\usepackage[textsize=tiny, textwidth=1.5cm]{todonotes} %add disable to options to not show in pdf

\usepackage{geometry}
\geometry{
	nomarginpar, % Toglie doppi margini
	margin=1in, % Imposta i margini a 1 inch
}

% Comandi per numero lezioni automatico

\newcounter{conteggioS}
\newcommand{\LezioneS}[1]{
	\stepcounter{conteggioS}
	\textit{Lezione Salsa \arabic{conteggioS} (#1)}
	}

\newcounter{conteggioV}
\newcommand{\LezioneV}[1]{
	\stepcounter{conteggioV}
	\textit{Lezione Verzini \arabic{conteggioV} (#1)}
	}


% Comandi pratici

% Lettere comuni in grassetto
\newcommand{\x}{\mathbf{x}}
\newcommand{\y}{\mathbf{y}}
\newcommand{\zer}{\mathbf{0}}

% Lettere comuni in grassetto greche
\usepackage{bm} % per avere le lettere greche in bold con \bm{\sigma}
\newcommand{\sigg}{\bm{\sigma}}
\newcommand{\nuu}{\bm{\nu}}
\newcommand{\alphaa}{\bm{\alpha}}

% d nell'integrale e i rispettivi usi
\newcommand{\de}{\,\mathrm d}
\newcommand{\dx}{\de x}
\newcommand{\dy}{\de y}
\newcommand{\dl}{\de l}
\newcommand{\dr}{\de r}
\newcommand{\ds}{\de s}
\newcommand{\dt}{\de t}
\newcommand{\dv}{\de v}
\newcommand{\dxi}{\de \xi}
\newcommand{\drho}{\de \rho}

% d nell'integrale con differenziale vettoriale
\newcommand{\dxx}{\de \x}
\newcommand{\dyy}{\de \y}
\newcommand{\dsig}{\de \sigg}

\allowdisplaybreaks[4] % Consente di rompere equazioni su più pagine

%%%%%%%%%%%%%%%%%%%%%%%%%%%%%%%%%%%%%%%%%%%%%%%
%%%%%%%%%%%%%%%%%%%%%%%%%%%%%%%%%%%%%%%%%%%%%%%

\begin{document}

%%%%%%%%%%%%%%%%%%%%%%%%%%%%%%%%%%%%%%%%%%%%%%%
%%%%%%%%%%%%%%%%%%%%%%%%%%%%%%%%%%%%%%%%%%%%%%%

\frontmatter
\pagestyle{empty}
\vspace*{\fill}
\begin{center}
	{\large \textsc{Lecture Notes of}}\\
	\vspace*{0.4cm}
	{\Huge \textsc{Model Identification}}\\
	\vspace*{0.4cm}
	{\Huge \textsc{and Data Analysis}}\\
	\vspace*{1cm}
	{\large {From Professor Simone Garatti's lectures}}\\
	\vspace*{0.1cm}
	{\large for the MSc in Mathematical Engineering}\\
	\vspace*{0.4cm}
	{\large {di Teo Bucci \& Filippo Cipriani}}\\
	\vspace*{1cm}
	Politecnico di Milano\\A.Y. 2021/2022
\end{center}
\vspace*{\fill}
\newpage

%%%%%%%%%%%%%%%%%%%%%%%%%%%%%%%%%%%%%%%%%%%%%%%
%%%%%%%%%%%%%%%%%%%%%%%%%%%%%%%%%%%%%%%%%%%%%%%

{\Large \textit{Lecture Notes of Model Identification and Data Analysis}}

\vspace*{\fill}

%\textcopyright \ Authors.

This text is provided under Creative Commons BY-NC-SA 4.0 license.\\
\url{https://creativecommons.org/licenses/by-nc-sa/4.0/}

The \LaTeX \ source code is available at\\
\url{https://github.com/teobucci/mida}

\vspace*{1cm}

Revisione del \today

Developed by\\
Teo Bucci - \texttt{teobucci8@gmail.com}\\
Filippo Cipriani - \texttt{filippo.cipriani99@hotmail.it}\\ \\
Compiled with \ensuremath\heartsuit \\

%\textbf{Prefazione}

Per segnalare eventuali errori o suggerimenti potete mandare una pull request.

\newpage

%%%%%%%%%%%%%%%%%%%%%%%%%%%%%%%%%%%%%%%%%%%%%%%
%%%%%%%%%%%%%%%%%%%%%%%%%%%%%%%%%%%%%%%%%%%%%%%

% INDICE
\addtocontents{toc}{\protect\thispagestyle{empty}}
\tableofcontents
%\newpage

%%%%%%%%%%%%%%%%%%%%%%%%%%%%%%%%%%%%%%%%%%%%%%%
%%%%%%%%%%%%%%%%%%%%%%%%%%%%%%%%%%%%%%%%%%%%%%%

% PAGINA VUOTA PER FAR PARTIRE IL CAPITOLO IN UNA PAGINA DISPARI
%\myNewEmptyPage

\AtEndDocument{\cleardoublepage}

%%%%%%%%%%%%%%%%%%%%%%%%%%%%%%%%%%%%%%%%%%%%%%%
%%%%%%%%%%%%%%%%%%%%%%%%%%%%%%%%%%%%%%%%%%%%%%%

\mainmatter
\pagestyle{fancy} % Riswitcha per riavere il numero pagina
%\setcounter{page}{1} % Fa ripartire il contatore pagina da 1

%%%%%%%%%%%%%%%%%%%%%%%%%%%%%%%%%%%%%%%%%%%%%%%
%%%%%%%%%%%%%%%%%%%%%%%%%%%%%%%%%%%%%%%%%%%%%%%

\part{MIDA I}

\chapter{Stochastic Processes}
A stochastic process (SP) is an infinite sequence of random variables all defined on the same probabilistic space:
\[
	\ldots,v(1,s),v(2,s),v(3,s),\ldots,v(t,s),\ldots
\]
with $s$: random experiment realization and $t=0,\pm 1,\pm 2,\ldots$: time index.

\textbf{Observation.} SP extends the notion of random vector (SP is a random vector with infinite entries).

\textbf{Observation.} For a fixed value of the random experiment $s = \overline{s}$, the stochastic process becomes the numeric sequence:
\[
	\ldots,v(1,\overline{s}),v(2,\overline{s}),v(3,\overline{s}),\ldots,v(t,\overline{s}),\ldots
\]
which is called \textbf{realization} of the stochastic process.
For different values of $s$, one gets different realizations of the stochastic process.

We will think of available observations ${u(1),u(2),\ldots,u(N)}$ and ${y(1), y(2),\ldots, y(N)}$ as finite length realizations of stochastic process (stochastic process = uncertain model for a time series or I/O system).

\textbf{Mean value} $m(t)$: it is the expected value of random variable $v(t,s)$ at time $t$:
\[
	m(t)=\mathbb{E}[v(t, s)]=\int v(t, s) \mathbb{P}(ds)
\]
$m(t)$ returns the value around which the process take value at time $t$.

\textbf{Covariance function} $\gamma(t_{1}, t_{2})$: it is the expected value of the product of unbiased random variables $(v(t, s)-m(t))$ at two time instants $(t_{1}, t_{2}):$
\begin{align*}
	\gamma(t_{1}, t_{2}) &=\mathbb{E}[(v(t_{1}, s)-m(t_{1}))(v(t_{2}, s)-m(t_{2}))] \\
	&=\int(v(t_{1}, s)-m(t_{1}))(v(t_{2}, s)-m(t_{2})) \mathbb{P}(ds)
\end{align*}
$\gamma(t_{1}, t_{2})$ quantifies the relation existing between the \emph{gaps} between the process realizations and the mean value at two different time instants.

Particular case: $t_{1}=t_{2}=t$.
\[
	\gamma(t, t)=\mathbb{E}[(v(t, s)-m(t))^{2}]=\int(v(t, s)-m(t))^{2} \mathbb{P}(ds)
\]
is called the process \textbf{variance function} (it quantifies the process dispersion around its mean value at each time instant).

\section{Stationary Stochastic Processes (SSP)}
A stochastic process is called stationary (wide sense) if:
\begin{itemize}
	\item $m(t)=m \quad \forall t$
	\item $\gamma(t_{1}, t_{2})$ depends on $\tau=t_{1}-t_{2}$ only, i.e.: $\gamma(t_{1}, t_{2})=\gamma(t_{3}, t_{4})$ if $t_{1}-t_{2}=t_{3}-t_{4}=\tau \quad \forall t_{1}, t_{2}, t_{3}, t_{4}$
\end{itemize}
Idea: the probabilistic properties of a SSP are time-translation invariant.

Stationary stochastic processes admit a \emph{simplified} representation of the covariance function:
\begin{gather*}
\gamma(\tau)=\gamma(t, t-\tau)=\mathbb{E}[(v(t)-m)(v(t-\tau)-m)]\qquad (t_{1}=t, t_{2}=t-\tau, t_{1}-t_{2}=\tau)
\end{gather*}
and
\[
	\gamma(0)=\mathbb{E}[(v(t)-m)^{2}]=\lambda^2  \qquad \text{is the variance of the process}
\]
Why stationary stochastic processes?
\begin{itemize}
	\item \emph{Stationary} means \emph{time-invariant} data generating system (situation often encountered in practice).
	\item S.S.P. are easier to study.
	\item Non-stationary processes can be recast in the framework of S.S.P. by first eliminating the non-stationary part from data (data pre-processing).
\end{itemize}

\textbf{Covariance function properties for an S.S.P.}
$$
\gamma(\tau)=\mathbb{E}[(v(t)-m)(v(t-\tau)-m)]
$$
\begin{itemize}
	\item $\gamma(0)=\mathbb{E}[(v(t)-m)^{2}] \geq 0$ (non negative at initial time)
	\item $|\gamma(\tau)| \leq \gamma(0)$ (bounded)
	\item $\gamma(\tau)=\gamma(-\tau)$ (symmetric) indeed
	\begin{align*}
		\gamma(-\tau)&=\mathbb{E}[(v(t)-m)(v(t-(-\tau))-m)]\\
		&=\mathbb{E}[(v(t)-m)(v(t+\tau))-m)]\\
		&=\mathbb{E}[(v(t+\tau)-m)(v(t))-m)]\\
		&=\gamma(\tau) \quad(t+\tau-t=\tau)
	\end{align*}
\end{itemize}

\fg{0.7}{Screen Shot 2022-03-06 at 00.50.16}

\textbf{Observation}
\begin{enumerate}
	\item Given a S.S.P. $x(t)$, we will write $m_{x}$ e $\gamma_{x}(\tau)$ for its mean and covariance function
	\item Two S.S.P. $y_{1}(t)$ and $y_{2}(t)$ are wide-sense equivalent if $m_{y_{1}}=m_{y_{2}}$ e $\gamma_{y_{1}}(\tau)=\gamma_{y_{2}}(\tau), \forall \tau$
	\item The \emph{covariance function}
	$$
		\mathbb{E}[(v(t)-m) \cdot(v(t-\tau)-m)]
	$$
	is very \emph{different} from the $2^{\text{nd}}$ order moment function $\mathbb{E}[v(t) \cdot v(t-\tau)]$.
\end{enumerate}

\textbf{Example 1}

$v(t,s)=\alpha (s),\forall t$ where $\alpha (s)\sim G(1,3)$ (Gaussian random variable with mean = 1 and variance = 3)

Is the process stationary? Yes:
\begin{itemize}
	\item $m_{v}(t)=\mathbb{E}[v(t, s)]=\mathbb{E}[\alpha(s)]=1=m_{v}$ doesn't depend on $t$.
	\item recalling that $v(t, s)=\alpha(s), \forall t$ and that $m_{v}(t)=1=m_{v}$
	\begin{align*}
		\gamma_{v}(t, t-\tau)&=\mathbb{E}[(v(t, s)-m_{v}(t))(v(t-\tau, s)-m_{v}(t-\tau))]\\
		&=\mathbb{E}[(\alpha(s)-1)(\alpha(s)-1)]=3=\gamma_{v}(\tau)
	\end{align*}
	 doesn't depend on $t$.
\end{itemize}

\textbf{Example 2}

$v(t, s)=t \cdot \alpha(s)-t$, where $\alpha(s) \sim G(1,3)$.
\begin{itemize}
	\item $m_{v}(t)=\mathbb{E}[v(t, s)]=\mathbb{E}[t \cdot \alpha(s)-t]=t \cdot \mathbb{E}[\alpha(s)]-t=t-t=0$ doesn't depend on $t$.
	\item \begin{align*}
		\gamma_{v}(t, t-\tau)&=\mathbb{E}[(v(t, s)-m_{v}(t))(v(t-\tau, s)-m_{v}(t-\tau))]\\
	&=\mathbb{E}[(t \cdot \alpha(s)-t)((t-\tau) \cdot \alpha(s)-(t-\tau))]\\
	&=\mathbb{E}[t \cdot(t-\tau)(\alpha(s)-1)^{2}]\\
	&=t \cdot(t-\tau) \cdot \mathbb{E}[(\alpha(s)-1)^{2}]=t \cdot(t-\tau) \cdot 3
	\end{align*}
	\emph{does} depend on $t$.
\end{itemize}
The process is not stationary.

\textbf{Observation.}

If $\gamma(t, \tau)>0$ then there is a tendency of preserving the sign of the deviation from $t$ to $\tau $. Otherwise there is a tendency of changing the sign.

\section{White Noise}

An S.S.P. $e(t)$ is called White Noise (WN) with mean $\mu$ and variance $\lambda^{2}$, we shall write
\[
	e(t) \sim W N(\mu, \lambda^{2}),
\]
if the following conditions hold:
\begin{itemize}
	\item $\mathbb{E}[e(t)]=\mu \quad \forall t$
	\item $\gamma_{e}(0)=\mathbb{E}[(e(t)-\mu)^{2}]=\lambda^{2} \quad \forall t$
	\item $\gamma_{e}(\tau)=\mathbb{E}[(e(t)-\mu) \cdot(e(t-\tau)-\mu)]=0 \quad \forall t, \forall \tau \neq 0$
\end{itemize}

The last property is the fundamental one. It says that there is complete incorrelation between random variables at different time instants. The realizations of $e(t)$ are erratic and unpredictable (\textbf{whiteness property}).

\fg{0.7}{Screen Shot 2022-03-08 at 09.53.35}

\textbf{Observation.} The probability distribution of each single random variables $e(t,s)$ does not matter and is not made explicit in general (wide-sense description of S.S.P.).
It could be Gaussian, uniform, etc. (WGN = White Gaussian Noise, WUN = White Uniform Noise, etc.).

\textbf{Observation.} Is a constant realization admissible? Yes, it is, but such a realization is \emph{highly unlikely}.

White Noise is a sort of \emph{building block} to construct a number of different stationary stochastic processes.

\textbf{Remark.} To ease the notation, in the following we will consider \emph{zero mean} white noise. The extension to the general case presents no conceptual difficulties.

\subsection{MA(\texorpdfstring{$n$}{n}) processes}

Read \emph{Moving Average of order $n$}.

Let $e(t) \sim W N(0, \lambda^{2})$, and MA process is obtained as
\[
	\boxed{y(t)=c_{0} e(t)+c_{1} e(t-1)+c_{2} e(t-2)+\ldots+c_{n} e(t-n).}
\]
In other words, the output $y(t)$ of a MA process is given by a linear combination of the last $n+1$ past values of the input noise $e(t)$.
While $t$ is let vary, the linear combination is made on a sliding window (moving average).

\textbf{Mean.}
\[
	m_{y}(t)=\mathbb{E}[y(t)] = \mathbb{E}[c_{0} e(t)+c_{1} e(t-1)+c_{2} e(t-2)+\cdots+c_{n} e(t-n)] = 0+0+\cdots+0=m_{y}=0 \quad \forall t
\]
hence $m_{y}(t)$ doesn't depend on $t$.

\textbf{Variance.} (i.e. covariance when $\tau =0$)
\begin{align*}
	\gamma (0)&=\mathbb{E}[(y(t)-m_{y})(y(t)-m_{y})]=\mathbb{E}[(y(t))^2]\\
	&=\mathbb{E}[(c_{0} e(t)+c_{1} e(t-1)+\ldots+c_{n} e(t-n))^{2}]\\
	&=\mathbb{E}[c_{0}^{2} e(t)^{2}+\ldots+c_{n}^{2} e(t-n)^{2}\\
	&\qquad+2 c_{0} c_{1} e(t) e(t-1)+\ldots+2 c_{n-1} c_{n} e(t-n-1) e(t-n)]\\
	&=c_{0}^{2} \mathbb{E}[e(t)^{2}]+c_{1}^{2} \mathbb{E}[e(t-1)^{2}]+\ldots+c_{n}^{2} \mathbb{E}[e(t-n)^{2}]\\
	&\qquad+2 c_{0} c_{1} \mathbb{E}[e(t) e(t-1)]+\ldots+2 c_{n-1} c_{n} \mathbb{E}[e(t-n-1) e(t-n)]
\end{align*}

Since $e(t) \sim W N(0, \lambda^{2})$, we have that:
$$
\mathbb{E}[e(t)^{2}]=\mathbb{E}[e(t-1)^{2}]=\ldots=\mathbb{E}[e(t-n)^{2}]=\lambda^{2}
$$
and that
$$
\mathbb{E}[e(t) e(t-1)]=\ldots=\mathbb{E}[e(t-n-1) e(t-n)]=0
$$
thus
\[
	\boxed{\gamma (t,t)=\gamma (0)=(c_{1}^2 +c_{1}^2 +\cdots+c_{n}^2 )\cdot\lambda^2}
\]
hence $\gamma (0)$ doesn't depend on $t$.

\textbf{Covariance.}

To calculate the generic covariance, let us proceed with $\tau =1$.
\begin{align*}
	\gamma(t, t-1)&=\mathbb{E}[(y(t)-m_{y})(y(t-1)-m_{y})]\\
	&=\mathbb{E}[y(t) y(t-1)]\\
	&=\mathbb{E}[(c_{0} e(t)+c_{1} e(t-1)+\ldots+c_{n} e(t-n))\cdot (y(t-1))]\\
	&=\mathbb{E}[(c_{0} e(t)+c_{1} e(t-1)+\ldots+c_{n} e(t-n))\cdot (c_{0} e(t-1)+\ldots+c_{n-1} e(t-n)+c_{n} e(t-n-1))]
\end{align*}

Only those terms where the white noise is multiplied by itself at the same time instant are non null.
\[
	\gamma(t, t-1)=\gamma (1)=(c_{0}c_{1}+c_{1}c_{2}+\cdots+c_{n-1}c_{n})\cdot\lambda^2
\]
hence $\gamma (1)$ doesn't depend on $t$.

Similarly
\begin{align*}
	\gamma (t,t-2)=\gamma (2) &= (c_{0}c_{2}+c_{1}c_{3}+\cdots+c_{n-2}c_{n})\cdot\lambda^2\\
	&\vdots\\
	\gamma (t,t-n)=\gamma (n) &= (c_{0}c_{n})\cdot\lambda^2\\
	\gamma (t,t-n-1)=\gamma (n+1) &= 0
\end{align*}
since all products are uncorrelated. In conclusion
\[
	\gamma (\tau )=\begin{cases}
		(c_{1}^2 +c_{1}^2 +\cdots+c_{n}^2 )\cdot\lambda^2 & \text{if}\ \tau =0\\
		(c_{0}c_{1}+c_{1}c_{2}+\cdots+c_{n-1}c_{n})\cdot\lambda^2 & \text{if}\ \tau =\pm 1\\
		(c_{0}c_{2}+c_{1}c_{3}+\cdots+c_{n-2}c_{n})\cdot\lambda^2 & \text{if}\ \tau =\pm 2\\
		\vdots\\
		(c_{0}c_{n})\cdot\lambda^2 & \text{if}\ \tau =\pm n\\
		0 & \text{if}\ \tau > \pm 1\\
	\end{cases}
\]
\subsection{MA(\texorpdfstring{$\infty$}{infinity}) processes}

\[
	y(t)=c_{0} e(t)+c_{1} e(t-1)+\cdots+c_{i} e(t-i)+\cdots=\sum_{i=0}^{\infty} c_{i} e(t-i) \quad e(t) \sim W N\left(0, \lambda^{2}\right)
\]
Assumption: $\sum_{i=0}^{\infty} c_{i}^{2}<\infty$ (it guarantees that $y(t)$ is well defined).

\textbf{Mean.}
\[
	m_{y}(t)=\mathrm{E}[y(t)]=\mathrm{E}\left[\sum_{i=0}^{\infty} c_{i} e(t-i)\right]=\sum_{i=0}^{\infty} c_{i} \mathrm{E}[e(t-i)]=\sum_{i=0}^{\infty} c_{i} \cdot 0=0
\]
doesn't depend on $t$.

\textbf{Variance.}
\begin{align*}
	\gamma_{y}(t, t)&=\mathrm{E}[(y(t)-m_{y})^{2}]\\
	&=\mathrm{E}\left[\sum_{i=0}^{\infty} c_{i} e(t-i) \cdot \sum_{j=0}^{\infty} c_{j} e(t-j)\right]\\
	&=\mathrm{E}\left[\sum_{i, j=0}^{\infty} c_{i} c_{j} \cdot e(t-i) e(t-j)\right]\\
	&=\sum_{i, j=0}^{\infty} c_{i} c_{j} \cdot \mathrm{E}[e(t-i) e(t-j)]\\
	&=\{\text{non null only when }i=j\}\\
	&=\sum_{i=0}^{\infty} c_{i}^2 \cdot\lambda^2 
\end{align*}
doesn't depend on $t$.

\textbf{Covariance.}
\begin{align*}
	\gamma_{y}(t, t-\tau) &=\mathrm{E}[(y(t)-m_{y}) (y(t-\tau)-m_{y})]\\
	&=\mathrm{E}[y(t) y(t-\tau)]\\
	&=\mathrm{E}\left[\sum_{i=0}^{\infty} c_{i} e(t-i) \cdot \sum_{j=0}^{\infty} c_{j} e(t-j-\tau)\right]\\
	&=\mathrm{E}\left[\sum_{i, j=0}^{\infty} c_{i} c_{j} \cdot e(t-i) e(t-j-\tau)\right]\\
	&=\sum_{i, j=0}^{\infty} c_{i} c_{j} \cdot \mathrm{E}[e(t-i) e(t-j-\tau)]\\
	&=\{\text{non null only when }i=j+\tau\}\\
	&=\sum_{i=0}^{\infty} c_{j+\tau}c_{j}\cdot\lambda^2 
\end{align*}
doesn't depend on $t$.

So if $\sum_{i=0}^{\infty} c_{i}^{2}<\infty$ then the MA($\infty $) process is well defined and is a S.S.P.

\textbf{Observation.} MA($\infty$) processes are very general, they almost \emph{cover} the class of stationary stochastic processes (i.e. apart from few exceptions, all S.S.P. can be written as MA($\infty$)).

However, MA($\infty$) are difficult to handle since there are infinite coefficients and, moreover, the computation of the covariance function requires the computation of the sum of an infinite series (hard in general).

On the other hand, MA($n$) are too limited, that is why we will look into AR and ARMA models.

\section{AR and ARMA processes}

\subsection{AR(Auto Regressive) processes}
A process $y(t)$ is an AR process if it is generated as:
\begin{align*}
	y(t)=a_{1} y(t-1)+a_{2} y(t-2)+\ldots+a_{m} y(t-m)+e(t)
\end{align*}
where $e(t) \approx W N\left(\mu, \lambda^{2}\right)$.

Terminology:

$\begin{array}{ll}a_{1}, a_{2}, \ldots, a_{m} & \text { AR process (model) coefficient } \\ m & \text { process (model) order; } \\ \operatorname{AR}(m) & \text { AR process of order } m .\end{array}$
 
Hence, the output $y(t)$ of an AR process is recursively defined as the linear combination of last $m$ past values of the process itself plus the input $e(t)$ at the same time instant.

\textbf{Observation.} The difference equation generating the AR process admits non-unique solution unless we specify an initial condition. Which solution do we consider as the AR process?

By AR process we mean the solution obtained by taking the initial condition $\mathbf{y}\left(t_{0}\right)=0$ and letting the initial time instant tends to minus infinity, $t_{0} \rightarrow-\infty$ (in short, we will write $\mathbf{y}(-\infty)=0$ )). In other words, the AR process is the steady-state solution.

\textbf{Example}

$y(t)=a y(t-1)+e(t)$, where $e(t) \approx W N\left(\mu, \lambda^{2}\right)(\operatorname{AR}(1)$ process $)$

What is the steady state solution?
$$
\begin{array}{rlr}
	y(t) & =a y(t-1)+e(t)= & (y(t-1)=a y(t-2)+e(t-1)) \\
	& =e(t)+a e(t-1)+a^{2} y(t-2)= & (y(t-2)=a y(t-3)+e(t-2)) \\
	& \cdots & \\
	& =e(t)+a e(t-1)+a^{2} e(t-2)+\ldots+a^{t-t_{0}} y\left(t_{0}\right)= & (y\left(t_{0}\right)=0) \\
	& \cdots & (t_{0} \rightarrow-\infty) \\
	& =e(t)+a e(t-1)+a^{2} e(t-2)+\ldots+a^{n} e(t-n)+\ldots=\sum_{i=0}^{\infty} a^{i} e(t-i)
\end{array}
$$
The steady state solution is an $\mathrm{MA}(\infty)$ process with coefficients: $c_{0}=1, c_{1}=a, c_{2}=a^{2}, \ldots, c_{i}=a^{i}, \ldots$

In general, AR processes are $\mathrm{MA}(\infty)$ processes with coefficients determined by the AR model coefficients by recursively apply the difference equation.

$\mathrm{MA}(\infty)$ processes are well defined if 
\begin{align*}
	\sum_{i=0}^{\infty} \left(c_{i}\right)^2=\sum_{i=0}^{\infty} \left(a^{i}\right)^2< +\infty
\end{align*}
The geometric series converges $\iff a^2\leq 1$.

So if $|a|\leq1$ the steady-state solution is well defined and SSP.

\subsection{ARMA(Auto Regressive Moving Average) processes}

A process $y(t)$ is an ARMA process if it is generated as:
\begin{align*}
	y(t)&=\\
	&=a_{1} y(t-1)+a_{2} y(t-2)+\ldots+a_{m} y(t-m)+\quad &\operatorname{AR}(m) \text{ part}\\ &+c_{0} e(t)+c_{1} e(t-1)+\ldots+c_{n} e(t-n) . \quad &\operatorname{MA}(n) \text{ part}
\end{align*}

where $e(t) \approx W N\left(\mu, \lambda^{2}\right)$.

Again by ARMA process we mean the steady-state solution obtained by letting $\mathbf{y}(-\infty)=0$.
 
 Similarly to AR processes, the steady-state solution is an MA( $\infty)$ process whose coefficients are obtained from the ARMA model coefficients by recursively apply the difference equation.
 
Terminology:

$m \quad$ AR part order

$n \quad$ MA part order

$\operatorname{ARMA}(m, n) \quad$ ARMA process of orders $m$ and $n$

An ARMA process is well defined and stationary under some conditions which are too complicated to verify directly.
We'll see how to solve this problem after introducing the operatorial 
representation of ARMA processes.

\subsection{Operatorial 
	representation of ARMA processes}

\textbf{Definition} (backward and forward shift operators)

The backward shift operator $z^{-1}$ (from the space of discrete-time signals to the same space) is defined as: $z^{-1} x(t)=x(t-1)$.

Similarly, $z$ is the forward shift operator and: $z x(t)=x(t+1)$.

\textbf{Properties of operators $z^{-1}$ and $z$}

$z^{-1}$ and $z$ are linear:

\begin{align*}
	&z^{-1}(a \cdot x(t)+b \cdot y(t))=a \cdot x(t-1)+b \cdot y(t-1) \\
	&z(a \cdot x(t)+b \cdot y(t))=a \cdot x(t+1)+b \cdot y(t+1)
\end{align*}

$z^{-1}$ and $z$ can be recursively applied:

\begin{align*}
	&z^{-1}\left(z^{-1}\left(z^{-1}(x(t))\right)\right)= \\
	&=z^{-1}\left(z^{-1}(x(t-1))\right)=z^{-1}(x(t-2))=x(t-3)= \\
	&=z^{-3} x(t) \text { (compact notation) }
\end{align*}

(similarly for $z$ )

$z^{-1}$ and $z$ can be linearly composed:

\begin{align*}
	&\left(a z^{-1}+b z+c z^{-3}+d z^{2}\right) x(t)= \\
	&=a\left(z^{-1} x(t)\right)+b(z x(t))+c\left(z^{-3} x(t)\right)+d\left(z^{2} x(t)\right)= \\
	&=a x(t-1)+b x(t+2)+c x(t-3)+d x(t+2)
\end{align*}

We can rewrite an ARMA process as follows:
\begin{align*}
	\left(1-a_{1} z^{-1}-a_{2} z^{-2}-\ldots-a_{m} z^{-m}\right) y(t)=\left(c_{0}+c_{1} z^{-1}+\ldots+c_{n} z^{-n}\right) e(t)
\end{align*}

Even more compact notation:
\begin{align*}
	y(t)=\frac{\left(c_{0}+c_{1} z^{-1}+\ldots+c_{n} z^{-n}\right)}{\left(1-a_{1} z^{-1}-a_{2} z^{-2}-\ldots-a_{m} z^{-m}\right)} e(t)=\frac{C(z)}{A(z)} e(t)
\end{align*}
where:
\begin{align*}
	&C(z)=\left(c_{0}+c_{1} z^{-1}+\ldots+c_{n} z^{-n}\right) \\
	&A(z)=\left(1-a_{1} z^{-1}-a_{2} z^{-2}-\ldots-a_{m} z^{-m}\right)
\end{align*}

$\frac{C(z)}{A(z)}$ is called discrete time transfer function and it simply says that $y(t)$ is generated as the steady-state output of a linear operator that receive as input $e(t)$.













































% lecture 01/03/2022
Say that $y(t)=W(z)u(t)$, where $W(z)=\frac{C(z)}{A(z)}$ means that $y(t)$ is the steady state solution to the recursive equation:
\[
	A(z)y(t)=C(z)u(t)
\]

\section{Composition of transfer functions and output processes}
\subsection{Series}
Given $u(t)$ stochastic process, consider
\begin{align*}
	x(t)&=W_{1}(z)u(t)=\frac{C_{1}(z)}{A_{1}(z)}u(t)\\
	y(t)&=W_{2}(z)x(t)=\frac{C_{2}(z)}{A_{2}(z)}x(t)
\end{align*}
\fg{0.7}{Screen Shot 2022-03-08 at 16.28.23}
\begin{theorem}
	The process $y(t)$ is the steady state output of a new filter having transfer function $W_{1}(z)\cdot W_{2}(z)$ fed by $u(t)$. That is,
	\[
		y(t)=[W_{1}(z)\cdot W_{2}(z)]u(t)=\frac{C_{1}(z)\cdot C_{2}(z)}{A_{1}(z)\cdot A_{2}(z)}u(t)
	\]
	meaning that $y(t)$ is the solution to the recursive equation $A_{1}(z)\cdot A_{2}(z) y(t) = C_{1}(z)\cdot C_{2}(z)u(t)$.
\end{theorem}

\subsection{Parallel}
Given $u(t)$ stochastic process, consider
\begin{align*}
	y_{1}(t)&=W_{1}(z)u(t)\\
	y_{2}(t)&=W_{2}(z)u(t)\\
	y(t)&=y_{1}(t)+y_{2}(t)=W_{1}(z)u(t)+W_{2}(z)u(t)=[W_{1}(z)+W_{2}(z)]u(t)
\end{align*}
\fg{0.7}{Screen Shot 2022-03-08 at 16.36.03}
\begin{theorem}
	The process $y(t)$ is the steady state output of a new filter having transfer function $W_{1}(z)+W_{2}(z)$ fed by $u(t)$.
\end{theorem}

\section{Poles and zeros}

\textbf{Remark.}
One can always multiply a transfer function by $z^{m}/z^{m}$.

Consider now $W(z)$ a complex-valued transfer function. Then one can identify:
\begin{itemize}
	\item \textbf{zeros}: all $z\in \mathbb{C}$ such that $W(z)=0$.
	\item \textbf{poles}: all $z\in \mathbb{C}$ such that $W^{-1} (z)=0$.
\end{itemize}
When $C(z),A(z)$ are polynomials with positive powers, then
\[
	\text{zeros}=\{z:(C(z)=0\} \qquad \text{poles}=\{z:(A(z)=0\}
\]

\textbf{Example.}
\begin{align*}
y(t) &=e(t)+\frac{1}{2} e(t-1)+\frac{1}{4} e(t-2) =\left(1+\frac{1}{2} z^{-1}+\frac{1}{4} z^{-2}\right) e(t) \\
&=\frac{1+\frac{1}{2} z^{-1}+\frac{1}{4} z^{-2}}{1} \cdot \frac{z^{2}}{z^{2}}\cdot e(t) =\frac{z^{2}+\frac{1}{2} z+\frac{1}{4}}{z^{2}} e(t)
\end{align*}
Poles are $z_{1}=z_{2}=0$.\\
Zeros are $z$ such that $z^{2}+\frac{1}{2} z+\frac{1}{4}=0$
\[
	z_{1,2}=\frac{-\frac{1}{2} \pm \sqrt{\left( \frac{1}{2}  \right) ^2 -4\cdot\frac{1}{4} } }{2} = -\frac{1}{4}\pm i\frac{\sqrt{3} }{4}
\]
\fg{0.7}{Screen Shot 2022-03-08 at 16.57.50}

\textbf{Remark.}
We say that $W(z)=C(z)/A(z)$ is
\begin{itemize}
	\item \textbf{asymptotically stable} if all \emph{poles} are such that $|z|<1$.
	\item \textbf{minimum phase} if all \emph{zeros} are such that $|z|<1$.
\end{itemize}

When ARMA processes are well-defined? Consider $W(z)$ rational transfer function, $v(t)$ stochastic process (input), $y(t)=W(z)v(t)$.

\begin{theorem}
	If
	\begin{itemize}
		\item $v(t)$ is a \emph{stationary} stochastic process;
		\item $W(z)$ is asymptotically stable;
	\end{itemize}
	then $y(t)$ is well-defined and is \emph{stationary}.
\end{theorem}

In the case of ARMA processes, the input is a White Noise, which is stationary by definition. Thus we just need to check the second condition.

In general one can factorize the denominator:
\begin{align*}
	y(t)&=\frac{C(z)}{A(z)}v(t)=\frac{C(z)}{(z-p_{1})(z-p_{2})\cdots(z-p_{m})}v(t)\\
	&=C(z)\cdot\frac{1}{z-p_{1}}\cdot\frac{1}{z-p_{2}}\cdots\frac{1}{z-p_{m}}v(t)\\
	&=\frac{1}{z-p_{m}}\left[ \frac{1}{z-p_{m-1}}\left[ \cdots\frac{1}{z-p_{1}}\left[ C(z)v(t) \right]   \right]   \right]  
\end{align*}
\fg{0.7}{Screen Shot 2022-03-08 at 17.10.36}






\end{document}