%!TEX root = ../main.tex

\begin{theorem}[Spectral Factorization]
	Let $y(t)$ be a \gls{ssp} with rational spectral density. 
	Then, there exists an unique \gls{wn} process $\xi(t)$ with suitable mean and variance and an unique rational transfer function $W(z)$ such that:
	\begin{align*}
		y(t) = W(z)\xi(t)=\frac{C(z)}{A(z)}\xi(t)
	\end{align*}
    and:
    \begin{enumerate}
		\item $C(z)$ and $A(z)$ are monic (i.e. the coefficients of the maximum degree terms of $C(z)$ and $A(z)$ are equal to 1);
		\item $C(z)$ and $A(z)$ have null relative degree;
		\item $C(z)$ and $A(z)$ are coprime (i.e. they have no common factors);
		\item[4a.]the poles of $W(z)$ are such that $|z|< 1$;
		\item[4b.]the zeros of $W(z)$ are such that $|z|\leq 1$.
    \end{enumerate}
\end{theorem}

When all the four conditions above are satisfied, we will say that $y(t) = W(z)\xi(t)$ is a \textbf{canonical representation} of $y(t)$.

\begin{obs}
Conditions 1, 2, and 3 remove any ambiguity due to the process described in Case 1, Case 2, and Case 3, respectively. 

Condition 4a ensures that $W(z)$ is asymptotically stable so that $y(t)$ is well defined, while 4b removes any ambiguity due to the process described in Case 4.
\end{obs}
\begin{example}
We want to find the canonical representation of:
\begin{align*}
	y(t)=\frac{2+5z^{-1}+2z^{-2}}{z^2+\frac{5}{6}z+\frac{1}{6}}\cdot e(t)  \qquad e(t)\sim \WN(0,1)
\end{align*}
We first need to have the same degree and positive powers, which we can achieved by multiplying by a proper factor and consider a new \gls{wn}:
\begin{equation*}
	y(t)=
	\frac{2+5z^{-1}+2z^{-2}}{z^2+\frac{5}{6}z+\frac{1}{6}}\cdot e(t)
	=\frac{2+5z^{-1}+2z^{-2}}{z^2+\frac{5}{6}z+\frac{1}{6}}\cdot\frac{z^2}{2}\cdot\underbrace{\frac{2}{z^2}}_{\eta(t)}\cdot e(t)
	=\frac{z^2+\frac{5}{2}z+1}{z^2+\frac{5}{6}z+\frac{1}{6}}\cdot \eta(t)
\end{equation*}
where $\eta(t)=2z^{-2}e(t)=2e(t-2)\sim \WN(2\cdot0,4\cdot 1)\sim \WN(0,4)$.\\
We know check if some terms cancel out:
\begin{align*}
	&\text{zeros:}\quad z^2+\frac{5}{2}z+1=(z+2)\left( z+\frac{1}{2} \right)   \implies z_{1,2}=-2,-\frac{1}{2}\\
	&\text{poles:}\quad z^2+\frac{5}{6}z+\frac{1}{6}=\left( z+\frac{1}{3} \right)  \left( z+\frac{1}{2} \right) \implies p_{1,2}=-\frac{1}{3},-\frac{1}{2}
\end{align*}
Then:
\begin{align*}
	y(t)=\frac{(z+2)\cancel{(z+\frac{1}{2})}}{(z+\frac{1}{3})\cancel{(z+\frac{1}{2})}}\cdot \eta(t) 
\end{align*}
We can now make sure that all zeros and poles are strictly inside the unit circle. In this case it is particularly useful to remember the all-pass filter:
\begin{align*}
	y(t)=\frac{z+2}{z+\frac{1}{3}}\cdot\frac{z+\frac{1}{2} }{z+\frac{1}{2}}\cdot\eta(t)=\frac{z+\frac{1}{2}}{z+\frac{1}{3}}\underbrace{\frac{z+2}{(z+\frac{1}{2})}\eta(t)}_{\xi(t)}=\frac{(z+\frac{1}{2})}{(z+\frac{1}{3})}\cdot \xi(t)
\end{align*}
where $\xi(t)\sim \WN(0,4\cdot2^2)\sim \WN(0,16)$. This is our canonical representation.
\end{example}
\chapter{Prediction}
\section{Optimal prediction for ARMA processes}
Let us consider a zero mean ARMA process:
\[
	y(t)=W(z)e(t) \qquad e(t)\sim \WN(0,\lambda^{2} )
\]

\textbf{Fundamental assumptions:}\label{assumptions-prediction-theory}
\begin{enumerate}
	\item $y(t) = W(z)e(t)$ is a canonical representation;
	\item $W(z)$ is minimum phase (i.e. all zeros are such that $|z|<1$).\footnote{Notice that \emph{canonical} implies that all zeros are such that $|z|\leq 1$, but it's not enough, we have to exclude zeros on the unit circle.}
\end{enumerate}
 
\begin{obs}
The first assumption is not much of a hurdle since every ARMA process admits a canonical representation, and if $y(t)$ is not given in its canonical representation one can easily reconstruct it. The reason why ask for it will be clear later. 

The second one is a limiting assumption (mild) that excludes models belonging to a small subclass that can be approximated by other ARMA. For example the process $y(t)=e(t)+e(t-1)$, which can be written as $y(t)=(1+z^{-1})e(t)$ has a zero in $-1$, but it can be approximated with $y(t)=(1+0.999z^{-1})e(t)$.
\end{obs}

Given the observations of the process $y(t)$ up to time $t$:
$$
	\ldots , y(t-101), y(t-100), \ldots , y(t-2), y(t-1), y(t)
$$
predict the future value of the process at time $t + k$.


A predictor is any function of the available information which is used to guess the future value of the process.

$\hat{y}(t + k \mid t) =$ predictor of $y(t + k)$ given the observations up to time $t$.

In general: 
$$\hat{y}(t + k \mid t) = f ( y(t), y(t-1), y(t-2),\ldots)$$
i.e. the predictor is \emph{any} function of the available observations of the process $y(t)$.

We will suppose to have all the observations from $-\infty$ up to $t$ (infinite sequence of observations).

We will \emph{restrict} ourselves to functions $f$ which are \emph{linear}:
\begin{align*}
	\hat{y}(t + k \mid t)=\sum_{i=0}^{\infty}\alpha_i y(t-i)
	\quad \text{identical to} \quad
	\hat{y}(t \mid t - k)=\sum_{i=0}^{\infty}\alpha_i y(t-k-i)
\end{align*}
where ${\{\alpha_i\}}_{i=0}^\infty$ are parameters to be selected in order to achieve the \emph{best} result.

\textbf{Intuitive Goal.} 
We want a predictor which is good with respect to $S$ (of the specific experiment).
\[
	\hat{y}(t + k \mid t)\approx y(t+k)
\]

\begin{definition}[\Gls{mse} of prediction]
	\[
		\E[(y(t+k,s)-\hat{y}(t+k\mid t,s))^2]
	\]
\end{definition}

\textbf{Goal:} choose $\alpha_0,\ldots,\alpha_i,\ldots$ in order to minimize the \gls{mse}.