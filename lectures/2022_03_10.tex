%!TEX root = ../main.tex
Let us compute the spectral density of $\tilde{\xi}(t)$:
\begin{align*}
	\Gamma_{\tilde{\xi}}(\omega) &=\frac{\left|e^{j \omega}-q\right|^{2}}{\left|e^{j \omega}-\frac{1}{q}\right|^{2}} \cdot \lambda^{2}=\frac{\left(e^{j \omega}-q\right)\left(e^{-j \omega}-q\right)}{\left(e^{j \omega}-\frac{1}{q}\right)\left(e^{-j \omega}-\frac{1}{q}\right)} \cdot \lambda^{2}=\\
	&=\frac{1+q^{2}-q\left(e^{j \omega}+e^{-j \omega}\right)}{1+\frac{1}{q^{2}}-\frac{1}{q}\left(e^{j \omega}+e^{-j \omega}\right)} \cdot \lambda^{2}=\frac{1+q^{2}-2 q \cos (\omega)}{1+\frac{1}{q^{2}}-\frac{2}{q} \cos (\omega)} \cdot \lambda^{2}=\\
	&=q^2\cdot\frac{1+\frac{1}{q^{2}}-\frac{2}{q} \cos (\omega)}{1+\frac{1}{q^{2}}-\frac{2}{q} \cos (\omega)}\cdot \lambda^{2}=q^2\cdot \lambda^{2}
\end{align*}
The spectral density is constant for all values of $\omega$ , so that $\tilde{\xi}(t)$ is a white noise!

Moreover:
$$\mathrm{E}[\tilde{\xi}(t)]=\frac{1-q}{1-\frac{1}{q}} \cdot \mathrm{E}[\xi(t)]=q \cdot \frac{1-\frac{1}{q}}{1-\frac{1}{q}} \cdot \mu=q \cdot \mu$$
Hence, $\tilde{\xi}(t) \sim W N\left(q \cdot \mu, q^2 \cdot \lambda^{2}\right)$ and $y(t)=\tilde{W}(z) \tilde{\xi}(t)$ is a new ARMA representation of the process $y(t)$ (note that $y(t)$ is always the same, it never changed).

\begin{theorem}{Spectral Factorization}
	Let $y(t)$ be a S.S.P. with rational spectral density. 
	Then, there exists an unique white noise process $\xi(t)$ with suitable 
	mean and variance and an unique rational transfer function $W (z)$ such 
	that:
	\begin{align*}
		y(t) = W (z)\xi (t)
	\end{align*}
    and, ($C(z)$ and $A(z)$ are the numerator and denominator of $W (z)$):
    \begin{itemize}
   \item $C(z)$ and $A(z)$ are monic (i.e. the coefficients of the maximum 
    	degree terms of $C(z)$ and $A(z)$ are equal to 1) 
    	\item $C(z)$ and $A(z)$ have null relative degree
    	\item $C(z)$ and $A(z)$ are coprime (i.e. they have no common factors) 
    	\item the absolute value of the poles and the zeroes of $W (z)$ is less than 
    	or equal to 1 (i.e. poles and zeroes are inside the unit circle)
    \end{itemize}
\end{theorem}

When all the four conditions above are satisfied, we will say that 
$y(t) = W (z)\xi(t)$ is a canonical representation of $y(t)$.

\textbf{Observation.} Conditions 1, 2, and 3 remove any ambiguity as due to the process 
described in Case 1, Case 2, and Case 3, respectively. 

The first part of Condition 4 assure that W (z) is asymptotically stable 
so that y(t) is well defined, while the second part removes any
ambiguity as due to the process described in Case 4.

%ci sarebbe da fare un esempio lungo da zero

\section{Linear Optimal prediction theory (for ARMA processes) }
Let us consider a zero mean ARMA process:
$$ y(t) =W (z) e(t)  \quad \text{where }e(t)\sim W N(0,\lambda^{2} )$$
$W(z)=\frac{C(z)}{A(z)}$ is an asymptotically stable rational transfer function $\implies$ $y(t)$ is well defined and a S.S.P.

\textbf{Further Assumptions:}
\begin{itemize}
	\item $y(t) =W (z) e(t)$ is a canonical representation.
	\item $W(z)$ has no zeroes on the unit circle boundary (i.e. no 
	zeroes with absolute value equal to 1)
\end{itemize}
 
\textbf{Observations:}
First assumption is not much a hurdle since every ARMA
process admits a canonical representation, and if y(t) is not given in 
its canonical representation one can easily reconstruct it. Why 
first assumption is required will be clear later. 

The second one is a limitating assumption(mild) that exclude models belonging to a small subclass that can be approximated by other ARMA.

\textbf{Problem}(k -step prediction) 

Given the observations of the process $y(t)$ up to time $t$:
$$
\ldots , y(t-101), y(t-100), \ldots , y(t-2), y(t-1), y(t)
$$
predict the future value of the process at time $t + k$.


A predictor is any function of the available information which is used 
to guess the future value of the process.

$\hat{y}(t + k | t) =$ predictor of $y(t + k)$ given the observations up to time $t$.

In general: 
$$\hat{y}(t + k | t) = f ( y(t), y(t-1), y(t-2),\ldots)$$
i.e. the predictor is any function of the available observations of the 
process $y(t)$.

\textbf{Abstraction.} we suppose to have the whole observations from $-\infty$ up to $t$ (infinite sequence of observations)

Then we can write:
\begin{align*}
	\hat{y}(t + k | t)=\sum_{i=0}^{\infty}\alpha_i y(t-i)
\end{align*}
where ${\{\alpha_i\}}_0^\infty$ are parameters to be selected in order to achieve the best result.

\textbf{Intuitive Goal.} 
We want a predictor which is good w.r.t $S$ (of the specific experiment).
$$\hat{y}(t + k | t)\approx y(t+k) $$

We need to quantify this intuitive closeness concept be rewritten in 
some rigorous mathematical notion of distance valid for random variables.

\begin{definition}{Mean square prediction error}
	
	$\mathbb{E}[(y(t+k,s)-\hat{y}(t+k|t,s))^2]$
\end{definition}

\textbf{Goal:} choose $\alpha_0,\ldots,\alpha_i,\ldots$ in order to minimize the MSE.