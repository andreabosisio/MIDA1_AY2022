%!TEX root = ../main.tex
\section{Optimal prediction of ARMAX processes}
\[
y(t)=\underbrace{\frac{B(z)}{A(z)} u(t-a)}_{\text{deterministic}}+
\underbrace{\frac{C(z)}{A(z)} e(t)}_{\text{stochastic}}\qquad e(t)\sim \WN(0,\lambda^2 )
\]
Without loss of generality NON zero mean can be always incorporated in $u$.

available information:
\begin{gather*}
	y(t),y(t-1),\ldots \\
	u(t),u(t-1),\ldots
\end{gather*}

\textbf{Hypothesis:}
\begin{itemize}
	\item $\frac{C(z)}{A(z)}e(t)$  is a canonical representation (otherwise compute it)
	\item either $u$ is completely known(pre deterministic signal) from $t=-\infty$ up to $t=+\infty$ or $d\geq k$ (delay bigger than prediction error).
\end{itemize}

Let 
$$z(t)=y(t)-\frac{B(z)}{A(z)} u(t-d)$$

Then, $z(t)=\frac{C(z)}{A(z)} e(t)$ i.e. it is an ARMA process s.t.:

$$\frac{C(z)}{A(z)}=E(z)+z^{-k} \frac{F(z)}{A(z)} \quad\text{($k$-steps division between $C(z)$ and $A(z)$)}
$$
Then:
$$
\hat{z}(t+k \mid t)=\frac{F(z)}{C(z)} z(t)
$$

<<<<<<< Updated upstream
$$y(t)=\frac{B(z)}{A(z)} u(t+k-d)+z(t+k)$$
The first part is deterministically 
known, and hence can 
be trivially predicted.
\begin{align*}
	\hat{y}(t+k \mid t)&=\frac{B(z)}{A(z)} u(t+k-d)+z(t+k \mid t) \\
	&=\frac{B(z)}{A(z)} u(t+k-d)+\frac{F(z)}{C(z)} z(t) \\
	& =\frac{B(z)}{A(z)} u(t+k-d)+\frac{F(z)}{C(z)}\left(y(t)-\frac{B(z)}{A(z)} \underbrace{u(t-d)}_{z^{-k}u(t+k-d)}\right) \\
	&=\frac{B(z)}{C(z)} \cdot\left(\underbrace{\frac{C(z)}{A(z)}-\frac{z^{-k} F(z)}{A(z)}}_{E(z)}\right) u(t+k-d)+\frac{F(z)}{C(z)} y(t)
\end{align*}
$$
\hat{y}(t+k \mid t) =\frac{B(z) E(z)}{C(z)} \underbrace{u(t+k-d)}_{\text{known}}+\frac{F(z)}{C(z)} y(t)
$$

\section{Model Identification}
Up to now, we considered (ARMA/ARMAX) models and studied their properties: covariance and spectrum computation, 
prediction.

But where does the model comes from?

Model identification: retrieve suitable model from experiments on 
the real system 

\fg{0.7}{Screenshot (25)}

Identification problem: define an automatic procedure to find a 
model for S based on available (input/output or time series) data 

\fg{0.7}{Screenshot (26)}

\subsection{Parametric model identification}
First we select a parametric model class $\mathcal{M}(\vartheta)$
($\vartheta$= parameters vector – each different $\vartheta$ corresponds to a 
different model).

For example:
$$
\mathcal{M}=\left\{y(t)=\frac{B(z, \vartheta)}{A(z, \vartheta)} u(t-d)+\frac{C(z, \vartheta)}{A(z, \vartheta)} e(t), \quad e(t) \sim WN\left(0, \lambda^{2}\right), \vartheta\in\Theta\right\}
$$
is the family of ARMAX models whose coefficients are polynomjals.
\begin{align*}
	A(z, \vartheta)&=1-a_{1}(\vartheta) z^{-1}-\cdots-a_{m}(\vartheta) z^{-m} \\
	B(z, \vartheta)&=b_{1}(\vartheta)+b_{2}(\vartheta) z^{-1}+\cdots+b_{p}(\vartheta) z^{-p}\\
	C(z, \vartheta)&=c_0(\vartheta)+c_{1}(\vartheta) z^{-1}+\cdots+c_{n}(\vartheta) z^{-n}
\end{align*}

We will talk of \textbf{black-box identification} when no knowledge 
on the system is available and the model structure must be found from 
data only.

$\vartheta$ is directly the vector of coefficients of $A(z, \vartheta),B(z, \vartheta),C(z, \vartheta)$.

$$
\vartheta=\left[
	a_{1}  \ldots a_{m},  b_{1} \ldots b_{p} ,c_{1} \ldots c_{n}
\right]^{T}
$$

In all other cases: \textbf{grey-box identification}

We try to incorporate some a priori information about the real $S$ in the model class. $\vartheta$ may have some physical interpretation.

$\underline{\text { Example }}$
$$
M(\vartheta): y(t)=\frac{b+b^{2} z^{-1}}{1-a z^{-1}} u(t-d)+\frac{1+a z^{-1}}{1-a z^{-1}} e(t)
$$
Here, the parameter vector is given by $\vartheta=\left[\begin{array}{l}a \\ b\end{array}\right]$ only.


\textbf{Observation:} $\lambda^2$ is a parameter too which needs to be identified, however, $\lambda^2$ is much less important than other parameters. So, we will indicate by $\vartheta$ the vector of “important” 
parameters and keep $\lambda^2$ aside.

$\Theta$ is the set of admissible values for the parameter vector $\vartheta$.

It incorporates a-priori information on the possible value for the 
parameters. In the blackbox case $\Theta$ is as free as possible.

As we will see, to perform identification we will rely on the theory of 
prediction. Hence, we will assume the following assumption:

For every $\vartheta \in \Theta$, the stochastic part of $M(\vartheta)$ (i.e. the part depending on the white noise $\left.e(t) \sim W N\left(0, \lambda^{2}\right)\right)$ is \textbf{canonical} and has \textbf{no zeroes on the unit circle}.

The requirement that there are no zeroes on the unit circle instead 
poses some limitations on the systems we can identify. However:
\begin{itemize}
	\item zeroes on the unit circle are not usually required to model the 
	behavior of a given system
	\item the behavior of models with zeroes on the unit circle can be 
	approximate by means of models with zeroes close to the unit 
	circle
\end{itemize} 

\subsection{PEM (Prediction Error Minimization) identification}
Paradigm to map data into a value of $\bar{\vartheta}$.
$$
D^N=\left\{\bar{y(1)},\ldots,\bar{y(N)},\bar{u(1)},\ldots,\bar{u(N)}\right\}
$$
$D^N$ is a finite sequence of real numbers, observation of $y$ and $u$ over some horizon. $N$ is the length of the dataset.

\fg{0.5}{Screenshot (27)}

How to compare $\bar{y(1)},\ldots,\bar{y(N)}$ with $y(1,s),\ldots,y(N,s)$?

\fg{0.7}{Screenshot (28)}

\textbf{Idea}(Prediction Error Minimization)

$M(\vartheta)\implies \hat{M}(\vartheta)$ from stochastic models to predictive models.

ARMAX in $\mathcal{M}$ for a given value of $\vartheta$:

\begin{align*}
	M(\vartheta):\quad y(t)&=\frac{B(z, \vartheta)}{A(z, \vartheta)} u(t-d)+\frac{C(z, \vartheta)}{A(z, \vartheta)} e(t)\\
	&\Downarrow\\
	\hat{M}(\vartheta): \quad \hat{y}(t \mid t-1) &=\frac{B(z,\vartheta) E(z,\vartheta)}{C(z,\vartheta)} u(t-d)+\frac{F(z,\vartheta)}{C(z,\vartheta)} y(t-1) \qquad \text{one step prediction}
\end{align*}

\fg{0.7}{Screenshot (29)}


=======
$$y(t)=\underbrace{\frac{B(z)}{A(z)} u(t+k-d)}_{\text{This part of the process 
		is deterministically 
		known, and hence can 
		be trivially predicted}} +z(t+k)$$
>>>>>>> Stashed changes
