%!TEX root = ../main.tex
\section{Spectrum of a \glsentrylong{ssp}}
Given $y(t)$ a \gls{ssp}, we can compute:
\begin{itemize}
	\item Fourier transform
	\[
		\Gamma _{y}(\omega)=\fou{\gamma _{y}(\tau )}=\sum_{\tau =-\infty }^{+\infty} \gamma _{y}(\tau ) e^{-j\omega \tau } \qquad \omega \in [0,\pi ]
	\]
	\item Anti-Fourier transform
	\[
		\gamma _{y}(\tau)=\ifou{\Gamma _{y}(\omega)}=\frac{1}{2\pi }\int_{-\pi }^{\pi } \Gamma _{y}(\omega )e^{j\omega \tau }d\omega  
	\]
\end{itemize}

\textbf{Observation.}
If $\tau =0$ we have
\[
	\gamma _{y}(0)=\E[(y(t)-m_{y})^2 ]=\frac{1}{2\pi }\int_{-\pi }^{\pi } \Gamma _{y}(\omega )d\omega 
\]

\textbf{Remark.}
There is bijective relationship between the spectrum and the covariance function. Namely to describe $y(t)$ one could use $m_{y}$ and $\gamma _{y}(\tau )$ or $m_{y}$ and $\Gamma _{y}(\omega)$. It gives the same information from a different perspective, but it's useful to understand some properties.

\subsection{An alternative interpretation of the spectral density}
Suppose that an \gls{ssp} $y(t)$ is filtered through an (ideal) pass-band filter:
\fg{0.7}{Screen Shot 2022-03-12 at 16.36.12}
where $\tilde{y}(t)$ is the filtered output process.
\begin{theorem}[Kinchine--Wiener]
	The spectral density at a fixed $\omega$ equals to the mean energy of process realizations frequency by frequency:
	\[
		\Gamma _{y}(\overline{\omega})=\lim_{\delta  \to 0} \gamma _{\tilde{y}}(0)
	\]
\end{theorem}

\emph{Example.}
The \gls{wn} spectral density is constant, i.e. \gls{wn} energy is \emph{equally-distributed} all over the frequency domain.
\fg{0.7}{Screen Shot 2022-03-12 at 16.41.55}

\emph{Example (AR($1$)).}
Consider $y(t)=ay(t-1)+e(t)$.
\begin{itemize}
	\item if $a>0$ the sign tends to be maintained
	\item if $a<0$ the sign tends to changed
\end{itemize}
The transfer function is $y(t)=\frac{1}{1-az^{-1} }e(t)$, then
\begin{align*}
	\Gamma _{y}(\omega )&=\left|\frac{1}{1-ae^{-j\omega } }\right|^2 \cdot\underbrace{\Gamma _{e}(\omega )}_{=\lambda^2 } = \frac{1}{1-ae^{-j\omega}}\cdot\frac{1}{1-ae^{j\omega}} \cdot\lambda^2\\
	&=\frac{1}{1+a^2 e^{-j\omega }\cdot e^{j\omega }-ae^{-j\omega}-ae^{j\omega}} \cdot \lambda^2\\
	&=\frac{1}{1+a^2 -a\left( \frac{e^{j\omega}+e^{-j\omega}}{2} \right)\cdot 2}\\
	&=\frac{1}{1+a^2-2a\cos (\omega)} \cdot\lambda^2
\end{align*}
Depending on the values of $a$ we have different behaviors.

If $a>0$ energy is concentrated at low frequencies
\fg{0.8}{Screen Shot 2022-03-12 at 16.59.06}
If $a<0$ energy is concentrated at larger frequencies
\fg{0.8}{Screen Shot 2022-03-12 at 16.58.25}

\emph{Example (ARMA).}
Consider $y(t)=\frac{C(z)}{A(z)}e(t)$
\[
	\Gamma _{y}(\omega )=\left|\frac{C(e^{j\omega})}{A(e^{j\omega})}\right|^2 \cdot\lambda^2 =\frac{C(e^{j\omega})}{A(e^{j\omega} )} \cdot \frac{C(e^{-j\omega})}{A(e^{-j\omega} )}\cdot\lambda^2 
\]
$y(t)$ has a \emph{rational} spectral density. The reverse il also true:
\[
	y(t) \text{ ARMA} \iff W(z) \text{ rational}
\]
Indeed we have the following result:
\begin{theorem}
	Let $y(t)$ be a \gls{ssp} with rational spectral density.
	Then, there exists a \gls{wn} process $\xi(t)$ with suitable mean and variance and a rational transfer function $W(z)$ such that:
	\[
		y(t)=W(z)\xi(t)
	\]
	i.e. $y(t)$ is an ARMA process.
\end{theorem}

However, the \emph{choice} of $\xi(t)$ and of $W(z))$ \emph{is not unique}.
The \emph{same} process $y(t)$ can be generated according to an infinite number of different ARMA models.

Let $y(t)=W(z)\xi(t)$, where $\xi(t)$ is a \gls{wn}. Our goal is show how to construct $\tilde{W}(z)$ and $\tilde{\xi}(t)$ such that $\tilde{W}(z)\neq W(z)$ and $\tilde{\xi}(t)\neq \xi(t)$, but $y(t)=\tilde W(z)\tilde\xi(t)$. Let us consider four different cases:
\begin{enumerate}
	\item 
	Let $\alpha \in \R$
	\[
		y(t)=W(z) \xi(t) = \underbrace{\frac{W(z)}{\alpha}}_{\tilde{W}(z)} \cdot \underbrace{\alpha \xi(t)}_{\tilde{\xi}(t)} = \tilde{W}(z)\tilde{\xi}(t)
	\]
	where they are still both respectively a rational transfer function and a \gls{wn}, indeed:
	\begin{align*}
		\E[\tilde{\xi}(t)]&=\E[\alpha \xi(t)]=\alpha \E[\xi(t)]=\alpha \mu\\
		\gamma_{\tilde{\xi}}(\tau)&=\E[(\alpha\xi(t)-\alpha\mu)(\alpha \xi(t-\tau)-\alpha \mu)]=\alpha^2 \E[(\xi(t)-\mu)( \xi(t-\tau)- \mu)] = \alpha^2 \gamma_{\xi}(\tau)
	\end{align*}

	\item 
	Let $n \in \N$
	\[
		y(t)=W(z) \xi(t)=W(z) \cdot \frac{z^{n}}{z^{n}} \xi(t)=\left[z^{n} \cdot W(z)\right] \xi(t-n)=\tilde{W}(z) \tilde{\xi}(t)
	\]
	where they are still both respectively a rational transfer function and a \gls{wn}, indeed:
	\begin{align*}
		\E[\tilde{\xi}(t)]&=\E[\xi(t-n)]=\mu \\
		\gamma_{\tilde{\xi}}(\tau)&=\E[(\xi(t-n)-\mu)(\xi(t-n-\tau)-\mu)]=\gamma_{\xi}(\tau)
	\end{align*}

	\item 
	Let $p \in \C$ such that $|p|<1$
	\[
		y(t)=W(z) \xi(t) = \underbrace{\left[W(z) \cdot \frac{z-p}{z-p}\right]}_{\tilde{W}(z)} \xi(t)=\tilde{W}(z) \xi(t)
	\]
	where they are still both respectively a rational transfer function and a \gls{wn}.

	\item 
	Let $q$ be a zero of $W(z)$ such that $|q|>1$, that is
	\[
		W(z)=W_{1}(z)\cdot(z-q)
	\]
	then
	\begin{align*}
		y(t)&=W(z) \xi(t)=W_{1}(z) \cdot(z-q) \xi(t)=\left[W_{1}(z) \cdot(z-q) \cdot \frac{z-\frac{1}{q}}{z-\frac{1}{q}}\right] \xi(t)= \\
		&=\underbrace{W_{1}(z)\left(z-\frac{1}{q}\right)}_{\tilde{W}(z)}\cdot\underbrace{\frac{z-q}{z-\frac{1}{q}} \xi(t)}_{\tilde{\xi}(t)} = \tilde{W}(z) \tilde{\xi}(t)
	\end{align*}
	Note that $\tilde{\xi}$ is well-defined and stationary since we multiplied $\xi(t)$ by an asymptotically stable transfer function ($|z|=\left|\frac{1}{q}\right|<1$).\\
	Moreover they are still both respectively a rational transfer function and a \gls{wn}, indeed let us compute the spectral density of $\tilde{\xi}(t)$
	\begin{align*}
		\Gamma_{\tilde\xi}(\omega) &=\frac{\left|e^{j \omega}-q\right|^{2}}{\left|e^{j \omega}-\frac{1}{q}\right|^{2}} \cdot \lambda^{2}=\frac{\left(e^{j \omega}-q\right)\left(e^{-j \omega}-q\right)}{\left(e^{j \omega}-\frac{1}{q}\right)\left(e^{-j \omega}-\frac{1}{q}\right)} \cdot \lambda^{2}\\
		&=\frac{1+q^{2}-q\left(e^{j \omega}+e^{-j \omega}\right)}{1+\frac{1}{q^{2}}-\frac{1}{q}\left(e^{j \omega}+e^{-j \omega}\right)} \cdot \lambda^{2}=\frac{1+q^{2}-2 q \cos (\omega)}{1+\frac{1}{q^{2}}-\frac{2}{q} \cos (\omega)} \cdot \lambda^{2}\\
		&=q^{2} \frac{\frac{1}{q^{2}}+1-\frac{2}{q} \cos (\omega)}{1+\frac{1}{q^{2}}-\frac{2}{q} \cos (\omega)} \cdot \lambda^{2}=q^{2} \cdot \lambda^{2}
	\end{align*}
	which is always constant and characterises a \gls{wn}; while
	\[
		\boxed{\tilde \xi(t) = \frac{z-q}{z-\frac{1}{q}} \xi(t)} \implies \tilde \xi(t+1)-\frac{1}{q}\tilde \xi(t)=\xi(t+1)-q\xi(t)
	\]
	taking $\E[\cdot]$ on both sides
	\[
		m_{\tilde\xi} - \frac{1}{q} m_{\tilde\xi} = \mu - q\mu \implies m_{\tilde\xi} = \frac{1-q}{1-\frac{1}{q}} \cdot \mu = -q \frac{\cancel{1-\frac{1}{q}}}{\cancel{1-\frac{1}{q}}}=  - q \cdot \mu
	\]
	so in the end we have $\boxed{\tilde\xi(t)\sim \WN(-q\mu,q^2 \lambda^2 )}$.

	The quantity
	\[
		\boxed{\frac{z-q}{z-\frac{1}{q}}}
	\]
	is known as \textbf{all-pass filter}.
\end{enumerate}