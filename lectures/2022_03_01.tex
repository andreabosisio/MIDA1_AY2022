%!TEX root = ../main.tex

\section{Composition of transfer functions and output processes}
\subsection{Series}
Given $u(t)$ \gls{sp}, consider
\begin{align*}
	x(t)&=W_{1}(z)u(t)=\frac{C_{1}(z)}{A_{1}(z)}u(t)\\
	y(t)&=W_{2}(z)x(t)=\frac{C_{2}(z)}{A_{2}(z)}x(t)
\end{align*}
%\fg{0.7}{Screen Shot 2022-03-08 at 16.28.23}
\begin{figure}[htpb]
	\centering
	\begin{tikzpicture}
		% place nodes
		\node [block] (w1) at (0,0) {$W_{1}(z)$};
		\node [block,right=2cm of w1] (w2) {$W_{2}(z)$};

		% connect nodes
		\draw [stealth-] (w1.west) -- ++(-2,0) node[midway,above] {$u(t)$};
		\draw [-stealth] (w1.east) -- (w2.west) node[midway,above] {$x(t)$};
		\draw [-stealth] (w2.east) -- ++(2,0)  node[midway,above] {$y(t)$};
	\end{tikzpicture}
\end{figure}
\FloatBarrier

\begin{theorem}
	The process $y(t)$ is the steady state output of a new filter having transfer function $W_{1}(z)\cdot W_{2}(z)$ fed by $u(t)$. That is,
	\[
		y(t)=[W_{1}(z)\cdot W_{2}(z)]u(t)=\frac{C_{1}(z)\cdot C_{2}(z)}{A_{1}(z)\cdot A_{2}(z)}u(t)
	\]
	meaning that $y(t)$ is the solution to the recursive equation:
	\[
		A_{1}(z)\cdot A_{2}(z)\cdot y(t) = C_{1}(z)\cdot C_{2}(z)\cdot u(t).
	\]
\end{theorem}

\subsection{Parallel}
Given $u(t)$ \gls{sp}, consider
\begin{align*}
	y_{1}(t)&=W_{1}(z)u(t)\\
	y_{2}(t)&=W_{2}(z)u(t)\\
	y(t)&=y_{1}(t)+y_{2}(t)=W_{1}(z)u(t)+W_{2}(z)u(t)=[W_{1}(z)+W_{2}(z)]u(t)
\end{align*}
%\fg{0.7}{Screen Shot 2022-03-08 at 16.36.03}
\begin{figure}[htpb]
	\centering
	\begin{tikzpicture}

		\node (input) at (0,0) {};
		\draw[-] (input.east) -- +(1,0)
		    node[midway,above] {$u(t)$}
		    node[at end] (inputR) {};
		
		% blocks
		\node[block, above right=0.5cm and 1cm of inputR] (w1) {$W_1(z)$};
		\node[block, below right=0.5cm and 1cm of inputR] (w2) {$W_2(z)$};
		
		% connect block with input
		\draw[-stealth] (inputR.center)|-(w1.west);
		\draw[-stealth] (inputR.center)|-(w2.west);
		
		% sum node
		\node[sum, right=3.5cm of inputR] (s) {};
		
		% connect sum node
		\draw[-stealth] (w1.east)-|(s.north)
		    node[near start, above] {$y_1(t)$}
		    node[very near end, right] {$+$}
		;
		\draw[-stealth] (w2.east)-|(s.south)
		    node[near start, below] {$y_2(t)$}
		    node[very near end, right] {$+$}
		;
		\draw[-stealth] (s.east) -- ++(1,0)
		    node[midway, above] {$y(t)$}
		;
	\end{tikzpicture}
\end{figure}
\FloatBarrier
\begin{theorem}
	The process $y(t)$ is the steady state output of a new filter having transfer function $W_{1}(z)+W_{2}(z)$ fed by $u(t)$.
\end{theorem}

\section{Poles and zeros}

\begin{obs}
	One can always multiply a transfer function by $z^{m}/z^{m}$.

	Consider a complex-valued transfer function $W(z)=C(z)/A(z)$. Then one can identify:
	\begin{itemize}
		\item \textbf{zeros}: all $z\in \mathbb{C}$ such that $W(z)=0$.
		\item \textbf{poles}: all $z\in \mathbb{C}$ such that $W^{-1} (z)=0$.
	\end{itemize}
	When $C(z),A(z)$ are polynomials with \emph{positive powers}, then
	\[
		\text{zeros}=\{z:C(z)=0\} \qquad \text{poles}=\{z:A(z)=0\}
	\]
\end{obs}
\begin{example}
\begin{align*}
y(t) &=e(t)+\frac{1}{2} e(t-1)+\frac{1}{4} e(t-2) =\left(1+\frac{1}{2} z^{-1}+\frac{1}{4} z^{-2}\right) e(t) \\
&=\frac{1+\frac{1}{2} z^{-1}+\frac{1}{4} z^{-2}}{1} \cdot \frac{z^{2}}{z^{2}}\cdot e(t) =\frac{z^{2}+\frac{1}{2} z+\frac{1}{4}}{z^{2}} e(t)
\end{align*}
Poles are $z_{1}=z_{2}=0$.\\
Zeros are $z$ such that $z^{2}+\frac{1}{2} z+\frac{1}{4}=0$
\[
	z_{1,2}=\frac{-\frac{1}{2} \pm \sqrt{\left( \frac{1}{2}  \right) ^2 -4\cdot\frac{1}{4} } }{2} = -\frac{1}{4}\pm i\frac{\sqrt{3} }{4}
\]
\end{example}
%\fg{0.7}{Screen Shot 2022-03-08 at 16.57.50}
\begin{figure}[htpb]
	\centering
	\begin{tikzpicture}[scale=0.5]

		% Axes:
		\draw [-stealth] (-5,0) -- (5,0) node [below right]  {$\Re$};
		\draw [-stealth] (0,-5) -- (0,5) node [left] {$\Im$};

		% pole
		\node[cross out,draw=black] at (0,0) {};

		\draw[dashed] (-1,-1.73) -- (-1,1.73) node[midway, below left]{$-\frac{1}{4}$};
		
		% zeros    
		\draw[dashed] (0,1.73) node[right] {$i \frac{\sqrt{3}}{4}$} --  (-1,1.73) node[solid, fill=white, circle,draw=black] {};
		\draw[dashed] (0,-1.73) node[right] {$-i \frac{\sqrt{3}}{4}$} --  (-1,-1.73) node[solid, fill=white, circle,draw=black] {};

		% unit circle
		\draw (0,0) circle (4);
		\node[below right] at (4,0) {$1$};
		\node[above right] at (0,4) {$i$};
	\end{tikzpicture}
\end{figure}
\FloatBarrier

\begin{definition}
	We say that $W(z)=\frac{C(z)}{A(z)}$ is:
	\begin{itemize}
		\item \textbf{asymptotically stable} if all \emph{poles} are such that $|z|<1$.
		\item \textbf{minimum phase} if all \emph{zeros} are such that $|z|<1$.
	\end{itemize}
\end{definition}

\begin{theorem}
	Consider a rational transfer function $W(z)$ fed by a \gls{sp} $v(t)$, $y(t)=W(z)v(t)$. If
	\begin{itemize}
		\item $v(t)$ is a \gls{ssp};
		\item $W(z)$ is asymptotically stable;
	\end{itemize}
	then $y(t)$ is well-defined and is also \emph{stationary} (is a \gls{ssp}).
\end{theorem}

In the case of ARMA processes, the input is a \gls{wn}, which is stationary by definition. Thus we just need to check the asymptotical stability.

In general one can factorize the denominator:
\begin{align*}
	y(t)&=\frac{C(z)}{A(z)}v(t)=\frac{C(z)}{(z-p_{1})(z-p_{2})\cdots(z-p_{m})}v(t)\\
	&=C(z)\cdot\frac{1}{z-p_{1}}\cdot\frac{1}{z-p_{2}}\cdots\frac{1}{z-p_{m}}v(t)\\
	&=\frac{1}{z-p_{m}}\left[ \frac{1}{z-p_{m-1}}\left[ \cdots\frac{1}{z-p_{1}}\left[ C(z)v(t) \right]   \right]   \right]  
\end{align*}
%\fg{0.7}{Screen Shot 2022-03-08 at 17.10.36}
\begin{figure}[htpb]
	\centering
	\begin{tikzpicture}

	% place nodes
	\node [block] (c) at (0,0) {$C(z)$};
	\node [block] (p1) at (2.5,0) {$\frac{1}{z-p_1}$};
	\node [block] (p2) at (5,0) {$\frac{1}{z-p_2}$};

	\node [block] (pm) at (7.5,0) {$\frac{1}{z-p_m}$};

	% connect nodes
	\draw[stealth-] (c) -- ++(-1.5,0) node[midway,above] {$v(t)$};

	\draw[-stealth]  (c.east) -- (p1.west) node[midway,above] {$x_1(t)$};
	\draw[-stealth] (p1.east) -- (p2.west) node[midway,above] {$x_2(t)$};

	\draw[-] (p2.east) -- ++(0.2,0) node[at end,right] {$\ldots$};
	\draw[stealth-] (pm.west) -- ++(-0.4,0);
	\draw[-stealth] (pm.east) -- ++(1,0) node[midway,above] {$y(t)$};

	\end{tikzpicture}
\end{figure}
\FloatBarrier